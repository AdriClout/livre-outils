% Options for packages loaded elsewhere
\PassOptionsToPackage{unicode}{hyperref}
\PassOptionsToPackage{hyphens}{url}
\PassOptionsToPackage{dvipsnames,svgnames,x11names}{xcolor}
%
\documentclass[
  letterpaper,
  DIV=11,
  numbers=noendperiod]{scrreprt}

\usepackage{amsmath,amssymb}
\usepackage{iftex}
\ifPDFTeX
  \usepackage[T1]{fontenc}
  \usepackage[utf8]{inputenc}
  \usepackage{textcomp} % provide euro and other symbols
\else % if luatex or xetex
  \usepackage{unicode-math}
  \defaultfontfeatures{Scale=MatchLowercase}
  \defaultfontfeatures[\rmfamily]{Ligatures=TeX,Scale=1}
\fi
\usepackage{lmodern}
\ifPDFTeX\else  
    % xetex/luatex font selection
\fi
% Use upquote if available, for straight quotes in verbatim environments
\IfFileExists{upquote.sty}{\usepackage{upquote}}{}
\IfFileExists{microtype.sty}{% use microtype if available
  \usepackage[]{microtype}
  \UseMicrotypeSet[protrusion]{basicmath} % disable protrusion for tt fonts
}{}
\makeatletter
\@ifundefined{KOMAClassName}{% if non-KOMA class
  \IfFileExists{parskip.sty}{%
    \usepackage{parskip}
  }{% else
    \setlength{\parindent}{0pt}
    \setlength{\parskip}{6pt plus 2pt minus 1pt}}
}{% if KOMA class
  \KOMAoptions{parskip=half}}
\makeatother
\usepackage{xcolor}
\setlength{\emergencystretch}{3em} % prevent overfull lines
\setcounter{secnumdepth}{5}
% Make \paragraph and \subparagraph free-standing
\ifx\paragraph\undefined\else
  \let\oldparagraph\paragraph
  \renewcommand{\paragraph}[1]{\oldparagraph{#1}\mbox{}}
\fi
\ifx\subparagraph\undefined\else
  \let\oldsubparagraph\subparagraph
  \renewcommand{\subparagraph}[1]{\oldsubparagraph{#1}\mbox{}}
\fi

\usepackage{color}
\usepackage{fancyvrb}
\newcommand{\VerbBar}{|}
\newcommand{\VERB}{\Verb[commandchars=\\\{\}]}
\DefineVerbatimEnvironment{Highlighting}{Verbatim}{commandchars=\\\{\}}
% Add ',fontsize=\small' for more characters per line
\usepackage{framed}
\definecolor{shadecolor}{RGB}{241,243,245}
\newenvironment{Shaded}{\begin{snugshade}}{\end{snugshade}}
\newcommand{\AlertTok}[1]{\textcolor[rgb]{0.68,0.00,0.00}{#1}}
\newcommand{\AnnotationTok}[1]{\textcolor[rgb]{0.37,0.37,0.37}{#1}}
\newcommand{\AttributeTok}[1]{\textcolor[rgb]{0.40,0.45,0.13}{#1}}
\newcommand{\BaseNTok}[1]{\textcolor[rgb]{0.68,0.00,0.00}{#1}}
\newcommand{\BuiltInTok}[1]{\textcolor[rgb]{0.00,0.23,0.31}{#1}}
\newcommand{\CharTok}[1]{\textcolor[rgb]{0.13,0.47,0.30}{#1}}
\newcommand{\CommentTok}[1]{\textcolor[rgb]{0.37,0.37,0.37}{#1}}
\newcommand{\CommentVarTok}[1]{\textcolor[rgb]{0.37,0.37,0.37}{\textit{#1}}}
\newcommand{\ConstantTok}[1]{\textcolor[rgb]{0.56,0.35,0.01}{#1}}
\newcommand{\ControlFlowTok}[1]{\textcolor[rgb]{0.00,0.23,0.31}{#1}}
\newcommand{\DataTypeTok}[1]{\textcolor[rgb]{0.68,0.00,0.00}{#1}}
\newcommand{\DecValTok}[1]{\textcolor[rgb]{0.68,0.00,0.00}{#1}}
\newcommand{\DocumentationTok}[1]{\textcolor[rgb]{0.37,0.37,0.37}{\textit{#1}}}
\newcommand{\ErrorTok}[1]{\textcolor[rgb]{0.68,0.00,0.00}{#1}}
\newcommand{\ExtensionTok}[1]{\textcolor[rgb]{0.00,0.23,0.31}{#1}}
\newcommand{\FloatTok}[1]{\textcolor[rgb]{0.68,0.00,0.00}{#1}}
\newcommand{\FunctionTok}[1]{\textcolor[rgb]{0.28,0.35,0.67}{#1}}
\newcommand{\ImportTok}[1]{\textcolor[rgb]{0.00,0.46,0.62}{#1}}
\newcommand{\InformationTok}[1]{\textcolor[rgb]{0.37,0.37,0.37}{#1}}
\newcommand{\KeywordTok}[1]{\textcolor[rgb]{0.00,0.23,0.31}{#1}}
\newcommand{\NormalTok}[1]{\textcolor[rgb]{0.00,0.23,0.31}{#1}}
\newcommand{\OperatorTok}[1]{\textcolor[rgb]{0.37,0.37,0.37}{#1}}
\newcommand{\OtherTok}[1]{\textcolor[rgb]{0.00,0.23,0.31}{#1}}
\newcommand{\PreprocessorTok}[1]{\textcolor[rgb]{0.68,0.00,0.00}{#1}}
\newcommand{\RegionMarkerTok}[1]{\textcolor[rgb]{0.00,0.23,0.31}{#1}}
\newcommand{\SpecialCharTok}[1]{\textcolor[rgb]{0.37,0.37,0.37}{#1}}
\newcommand{\SpecialStringTok}[1]{\textcolor[rgb]{0.13,0.47,0.30}{#1}}
\newcommand{\StringTok}[1]{\textcolor[rgb]{0.13,0.47,0.30}{#1}}
\newcommand{\VariableTok}[1]{\textcolor[rgb]{0.07,0.07,0.07}{#1}}
\newcommand{\VerbatimStringTok}[1]{\textcolor[rgb]{0.13,0.47,0.30}{#1}}
\newcommand{\WarningTok}[1]{\textcolor[rgb]{0.37,0.37,0.37}{\textit{#1}}}

\providecommand{\tightlist}{%
  \setlength{\itemsep}{0pt}\setlength{\parskip}{0pt}}\usepackage{longtable,booktabs,array}
\usepackage{calc} % for calculating minipage widths
% Correct order of tables after \paragraph or \subparagraph
\usepackage{etoolbox}
\makeatletter
\patchcmd\longtable{\par}{\if@noskipsec\mbox{}\fi\par}{}{}
\makeatother
% Allow footnotes in longtable head/foot
\IfFileExists{footnotehyper.sty}{\usepackage{footnotehyper}}{\usepackage{footnote}}
\makesavenoteenv{longtable}
\usepackage{graphicx}
\makeatletter
\def\maxwidth{\ifdim\Gin@nat@width>\linewidth\linewidth\else\Gin@nat@width\fi}
\def\maxheight{\ifdim\Gin@nat@height>\textheight\textheight\else\Gin@nat@height\fi}
\makeatother
% Scale images if necessary, so that they will not overflow the page
% margins by default, and it is still possible to overwrite the defaults
% using explicit options in \includegraphics[width, height, ...]{}
\setkeys{Gin}{width=\maxwidth,height=\maxheight,keepaspectratio}
% Set default figure placement to htbp
\makeatletter
\def\fps@figure{htbp}
\makeatother
\newlength{\cslhangindent}
\setlength{\cslhangindent}{1.5em}
\newlength{\csllabelwidth}
\setlength{\csllabelwidth}{3em}
\newlength{\cslentryspacingunit} % times entry-spacing
\setlength{\cslentryspacingunit}{\parskip}
\newenvironment{CSLReferences}[2] % #1 hanging-ident, #2 entry spacing
 {% don't indent paragraphs
  \setlength{\parindent}{0pt}
  % turn on hanging indent if param 1 is 1
  \ifodd #1
  \let\oldpar\par
  \def\par{\hangindent=\cslhangindent\oldpar}
  \fi
  % set entry spacing
  \setlength{\parskip}{#2\cslentryspacingunit}
 }%
 {}
\usepackage{calc}
\newcommand{\CSLBlock}[1]{#1\hfill\break}
\newcommand{\CSLLeftMargin}[1]{\parbox[t]{\csllabelwidth}{#1}}
\newcommand{\CSLRightInline}[1]{\parbox[t]{\linewidth - \csllabelwidth}{#1}\break}
\newcommand{\CSLIndent}[1]{\hspace{\cslhangindent}#1}

\KOMAoption{captions}{tableheading}
\makeatletter
\makeatother
\makeatletter
\@ifpackageloaded{bookmark}{}{\usepackage{bookmark}}
\makeatother
\makeatletter
\@ifpackageloaded{caption}{}{\usepackage{caption}}
\AtBeginDocument{%
\ifdefined\contentsname
  \renewcommand*\contentsname{Table of contents}
\else
  \newcommand\contentsname{Table of contents}
\fi
\ifdefined\listfigurename
  \renewcommand*\listfigurename{List of Figures}
\else
  \newcommand\listfigurename{List of Figures}
\fi
\ifdefined\listtablename
  \renewcommand*\listtablename{List of Tables}
\else
  \newcommand\listtablename{List of Tables}
\fi
\ifdefined\figurename
  \renewcommand*\figurename{Figure}
\else
  \newcommand\figurename{Figure}
\fi
\ifdefined\tablename
  \renewcommand*\tablename{Table}
\else
  \newcommand\tablename{Table}
\fi
}
\@ifpackageloaded{float}{}{\usepackage{float}}
\floatstyle{ruled}
\@ifundefined{c@chapter}{\newfloat{codelisting}{h}{lop}}{\newfloat{codelisting}{h}{lop}[chapter]}
\floatname{codelisting}{Listing}
\newcommand*\listoflistings{\listof{codelisting}{List of Listings}}
\makeatother
\makeatletter
\@ifpackageloaded{caption}{}{\usepackage{caption}}
\@ifpackageloaded{subcaption}{}{\usepackage{subcaption}}
\makeatother
\makeatletter
\@ifpackageloaded{tcolorbox}{}{\usepackage[skins,breakable]{tcolorbox}}
\makeatother
\makeatletter
\@ifundefined{shadecolor}{\definecolor{shadecolor}{rgb}{.97, .97, .97}}
\makeatother
\makeatletter
\makeatother
\makeatletter
\makeatother
\ifLuaTeX
  \usepackage{selnolig}  % disable illegal ligatures
\fi
\IfFileExists{bookmark.sty}{\usepackage{bookmark}}{\usepackage{hyperref}}
\IfFileExists{xurl.sty}{\usepackage{xurl}}{} % add URL line breaks if available
\urlstyle{same} % disable monospaced font for URLs
\hypersetup{
  pdftitle={Introduction aux sciences sociales numérique},
  pdfauthor={CLESSN},
  colorlinks=true,
  linkcolor={blue},
  filecolor={Maroon},
  citecolor={Blue},
  urlcolor={Blue},
  pdfcreator={LaTeX via pandoc}}

\title{Introduction aux sciences sociales numérique}
\author{CLESSN}
\date{2023-07-15}

\begin{document}
\maketitle
\ifdefined\Shaded\renewenvironment{Shaded}{\begin{tcolorbox}[frame hidden, enhanced, breakable, sharp corners, boxrule=0pt, interior hidden, borderline west={3pt}{0pt}{shadecolor}]}{\end{tcolorbox}}\fi

\renewcommand*\contentsname{Table of contents}
{
\hypersetup{linkcolor=}
\setcounter{tocdepth}{2}
\tableofcontents
}
\bookmarksetup{startatroot}

\hypertarget{avant-propos}{%
\chapter*{Avant-propos}\label{avant-propos}}
\addcontentsline{toc}{chapter}{Avant-propos}

\markboth{Avant-propos}{Avant-propos}

Ceci est un exemple de citation Adcock and Collier (2001) .

\bookmarksetup{startatroot}

\hypertarget{trois-duxe9fis-pour-une-contribution-aux-sciences-sociales-numuxe9riques}{%
\chapter{Trois défis pour une contribution aux sciences sociales
numériques}\label{trois-duxe9fis-pour-une-contribution-aux-sciences-sociales-numuxe9riques}}

Ce premier chapitre n'est sans doute pas le plus excitant. Il ne
comprend ni graphique ni exercice. Il s'ancre dans la réflexion
théorique plutôt que dans la pratique méthodologique. Habituellement,
c'est la partie que l'on ignore, celle que l'on saute pour passer aux «
choses sérieuses ». Amateur de « choses sérieuses »? Bonne nouvelle! Cet
ouvrage en est rempli. Tout comme la carrière qui s'offre à vous si vous
choisissez de poursuivre dans l'étude des sciences sociales numériques.

En 2020, le monde est numérique, et rien ne semble présager un
inversement de la tendance. Au contraire, celle-ci risque plutôt de
s'accélérer. La pandémie de la COVID-19 a offert quelques-uns des
meilleurs exemples de cette tendance: télétravail généralisé, école
numérique, livraison en ligne, mobilisation via les réseaux sociaux,
intelligences artificielles pour le dépistage de fausses nouvelles et
application mobile pour tracer les déplacements et freiner les
pandémies. L'avenir est au numérique. Pour les jeunes chercheurs en
sciences sociales, cela équivaut à une montagne de «choses sérieuses».

Dans ce contexte, il ne fait aucun doute que votre carrière sera
passionnante. Si vous n'en êtes pas déjà convaincu, ce livre vous
fournira une panoplie d'exemples de vos nombreuses possibilités. De
l'analyse textuelle dans les médias aux sondages en ligne de milliers
d'individus, en passant par l'extraction de données massives des sites
web ou à l'analyse de larges réseaux de communication, vous trouverez
assurément des défis à la hauteur de vos aspirations.

Devant ce déluge de données numériques, le jeune chercheur peut avoir
l'impression qu'il est possible, voire permis de tout faire.
Entendons-nous bien: c'est presque le cas. Tous les jours, vous aurez
des idées de projets plus invraisemblables les unes que les autres. Avec
vos nouveaux outils, plusieurs de ces idées n'auront aucun problème à se
réaliser. Le véritable problème surviendra peut-être le jour où sera
négligée la réflexion théorique. La réflexion au cœur même de ce
chapitre. Rappelez-vous: c'est ce chapitre que vous avez considéré
sauter, au départ!

En fait, il serait surprenant que vous ne soyez pas happés, très tôt
dans vos études, à des limites fondamentales à votre travail. Dans cet
ouvrage, nous les appelleront « défis ». Nous ne parlons pas ici de
données manquantes ou d'accès restreints à l'information. Il s'agit de
défis beaucoup plus élémentaires. Ils se comptent au nombre de trois et
sont à la base de toute réflexion préalable à la recherche en science
sociales numériques. Ils sont:

\begin{enumerate}
\def\labelenumi{\arabic{enumi}.}
\tightlist
\item
  Le défi technique;
\item
  Le défi théorique;
\item
  Le défi éthique.
\end{enumerate}

Sachez une chose: ces défis sont présents dans toutes les grandes
branches de la science, c'est-à-dire lors de la recherche, lors de la
diffusion des résultats et lors de l'enseignement. Que vous comptiez
opérer dans l'une, dans l'autre ou dans toutes ces branches, une bonne
compréhension des trois défis permettra de limiter les risques d'impair,
mais surtout d'élargir l'univers de vos possibles.

\hypertarget{duxe9fi-1-linuxe9vitable-duxe9fi-technique.}{%
\section{Défi \#1: l'inévitable défi
technique.}\label{duxe9fi-1-linuxe9vitable-duxe9fi-technique.}}

Le premier défi est technique, lié à l'extraction et à l'analyse des
données numériques. Il nécessite l'apprentissage et le développement des
méthodologies. Avec R dans sa poche, ce défi est hautement simplifié. R
permet de penser autrement les possibilités de recherche, et de
travailler avec des outils tels que \emph{Shiny}, pour la création
instantanée d'applications web interactives ou \emph{Mechanical Turk},
pour la mise en ligne de micro-tâches (\emph{crowdsourcing}) à réaliser
à faible coût par des volontaires. R facilite également la réalisation
de revues de la portée de la littérature (\emph{scoping review}), une
technique permettant de cartographier la littérature scientifique dans
un champ donné.

Les données massives nous entourent. Que ce soit au travers de sondages,
via les médias sociaux ou à l'intérieur des archives gouvernementales en
ligne, il est plus facile que jamais de rassembler de grandes quantités
d'information. Le défi demeure toutefois complexe lorsque vient le temps
d'extraire et d'analyser ces données afin de contribuer à la
connaissance scientifique.

Déjà, cet ouvrage offre une base solide sur laquelle développer vos
méthodes. Celles-ci sont de plus en plus simples à apprendre et à
appliquer, notamment grâce aux réseaux de collaboration en ligne.
Aujourd'hui, une question peut rapidement être répondue après une
recherche sur Google. \emph{Stack Overflow} est un site web dédié à
l'entraide entre programmeurs. Vous le trouverez hautement utile.

Si les méthodologies sont plus efficaces que jamais, beaucoup restent
encore à faire pour permettre la transparence et l'accessibilité des
données publiques, la collaboration entre chercheurs et l'optimisation
des outils d'extractions de données. Le cœur du défi technique réside
dans l'amélioration des outils utiles et nécessaires aux chercheurs.

En effet, après l'apprentissage des méthodes disponibles à l'heure
actuelle, vous pourrez rapidement contribuer à leur optimisation. La
beauté d'un logiciel libre comme R est qu'il est possible pour tous de
développer de nouvelle manière de facilité la recherche, et d'ensuite
partager ces trouvailles avec le monde entier. Sur R, vous pourrez
construire des fonctions qui accéléreront votre travail. Le
développement de quatre ou cinq fonctions pourrait ensuite faire l'objet
d'un tout nouveau «package», que vous partagerez en ligne.

Tous les jours, de nouveaux packages R sont développés et mis en ligne.
Des dizaines existent simplement pour réaliser de l'analyse textuelle
automatisée, par exemple, une méthodologie qui permet l'étude
quantitative de large corpus de textes. Plusieurs de ces packages, comme
«Quanteda», «Topicmodels», ou ceux de la «Tidyverse» sont hautement
performants, et en constante amélioration.

Il est à la portée de toute chercheuse et de tout chercheur de
participer à la bonification des outils et à l'avancement des
méthodologies. C'est la réponse attendu au défi technique.

\hypertarget{duxe9fi-2-le-nuxe9cessaire-duxe9fi-thuxe9orique.}{%
\section{Défi \#2: le nécessaire défi
théorique.}\label{duxe9fi-2-le-nuxe9cessaire-duxe9fi-thuxe9orique.}}

\begin{itemize}
\tightlist
\item
  Nécessite une formation selon les principaux travaux scientifiques qui
  étudient l'impact des données numériques sur les théories en sciences
  sociales:

  \begin{itemize}
  \tightlist
  \item
    On ne doit pas réinventer la roue à chaque article scientifique;
  \item
    Comment intégrer nos travaux à la littérature actuelle?;
  \item
    Comment faire progresser cette littérature? Démontrer l'impact des
    données numériques sur les théories existantes;
  \item
    Exemple: le nationalisme: peut-on mesurer le nationalisme au travers
    des médias sociaux? Si oui, comment cela peut-il contribuer à la
    littérature sur le nationalisme?
  \end{itemize}
\end{itemize}

\hypertarget{duxe9fi-3-luxe9pineux-duxe9fi-uxe9thique.}{%
\section{Défi \#3: l'épineux défi
éthique.}\label{duxe9fi-3-luxe9pineux-duxe9fi-uxe9thique.}}

\begin{itemize}
\tightlist
\item
  Autour des questions de l'effet de l'ère numérique sur la
  confidentialité, la sécurité informatique, le consentement et le droit
  des sujets secondaires:

  \begin{itemize}
  \tightlist
  \item
    Le numérique offre beaucoup d'opportunité tout à fait légale, mais
    pas nécessairement éthique;
  \item
    Nécessaire d'encourager la réflexion par rapport aux défis humains
    entourant l'utilisation des nouvelles données numériques;
  \item
    Comment utiliser ces données pour améliorer les vies, sans brimer
    les libertés individuelles?;
  \item
    Exemple: intelligence artificielle (machine learning): Le milieu
    académique est loin d'être seul à s'intéresser à la grande quantité
    d'information disponible. Les partis politiques, les agences de
    marketing et bien d'autres organisations utilisent ces informations
    à des fins de victoires, ou de ventes.
  \end{itemize}
\end{itemize}

\hypertarget{conclusion-du-chapitre}{%
\section{Conclusion du chapitre}\label{conclusion-du-chapitre}}

\begin{itemize}
\tightlist
\item
  Au travers des nouveaux apprentissages et des exemples qui sont
  offerts dans ce livre, le lecteur est encouragé à se poser ces 3
  questions:
\end{itemize}

\begin{enumerate}
\def\labelenumi{\arabic{enumi}.}
\tightlist
\item
  D'abord, comment puis-je utiliser ces nouveaux outils pour faire
  progresser les méthodologies de recherche actuelles?
\item
  Ensuite, comment puis-je utiliser ces nouveaux outils pour contribuer
  à l'avancement des théories de mes champs de recherche?
\item
  Enfin, comment puis-je utiliser ces nouveaux outils pour exercer un
  impact positif sur mes semblables?
\end{enumerate}

\bookmarksetup{startatroot}

\hypertarget{section}{%
\chapter{}\label{section}}

\bookmarksetup{startatroot}

\hypertarget{section-1}{%
\chapter{}\label{section-1}}

\bookmarksetup{startatroot}

\hypertarget{le-monde-du-libre}{%
\chapter{Le monde du libre}\label{le-monde-du-libre}}

\emph{« Vous n'avez pas à suivre une recette avec précision. Vous pouvez
laisser de côté certains ingrédients. Ajouter quelques champignons parce
que vous en raffolez. Mettre moins de sel car votre médecin vous le
conseille --- peu importe. De surcroît, logiciels et recettes sont
faciles à partager. En donnant une recette à un invité, un cuisinier n'y
perd que du temps et le coût du papier sur lequel il l'inscrit. Partager
un logiciel nécessite encore moins, habituellement quelques clics de
souris et un minimum d'électricité. Dans tous les cas, la personne qui
donne l'information y gagne deux choses : davantage d'amitié et la
possibilité de récupérer en retour d'autres recettes intéressantes. »} -
Richard Stallman

Cette analogie illustre bien trois concepts au coeur de la philosophie
de Richard Stallman, souvent considéré comme le père fondateur du
logiciel libre: liberté, égalité, fraternité. Les utilisateurs de ces
logiciels sont libres, égaux, et doivent s'encourager mutuellement à
contribuer à la communauté. Ainsi, un logiciel libre est généralement le
fruit d'une collaboration entre développeurs qui peuvent provenir des
quatre coins du globe. Une réflexion éthique est au coeur du mouvement
du logiciel libre, dont les militants font campagne pour la liberté des
utilisateurs dès le début des années 1980. La Free Software Foundation
(FSF), fondée par Richard Stallman en 1985, définit rapidement le
logiciel «libre» {[}free{]} comme garant de quatre libertés
fondamentales de l'utilisateur: la liberté d'utiliser le logiciel sans
restrictions, la liberté de le copier, la liberté de l'étudier, puis la
liberté de le modifier pour l'adapter à ses besoins puis le redistribuer
\footnote{La redistribution doit évidemment respecter certaines
  conditions précises, dont l'enfreint peut mener à des condamnations
  {[}http://www.softwarefreedom.org/resources/2008/shareware.html{]}.}
Il s'agit ainsi d'un logiciel dont le code source\footnote{Pour rester
  dans les analogies culinaires, le code source est au logiciel est ce
  que la recette est à un plat: elle indique les actions à effectuer,
  une par une, pour arriver à un résultat précis. Encore une fois, cette
  dernière peut-être adaptée, modifiée, bonifiée.} est disponible, afin
de permettre aux internautes de l'utiliser tel quel ou de le modifier à
leur guise. Puisque le langage machine est difficilement lisible par
l'homme et rend la compréhension du logiciel extrêmement complexe,
l'accès au code source devient essentiel afin de permettre à
l'utilisateur de savoir ce que le fait programme fait réellement.
Seulement de cette façon, l'utilisateur peut \emph{contrôler} le
logiciel, plutôt que de se faire contrôler par ce dernier (Stallman,
1986).

\hypertarget{uxe9mergence-et-ascension}{%
\section{Émergence et ascension}\label{uxe9mergence-et-ascension}}

Plusieurs situent les débuts du mouvement du logiciel libre avec la
création de la licence publique générale GNU\footnote{expliquer ce
  qu'est GNU en quelques lignes/le modèle collaboratif de développement
  logiciel initié par le projet GNU}, en 1983, à partir de laquelle va
se développer une multitude de programmes libres. Depuis, la popularité
des logiciels libres n'a cessé de croître, alors que des dizaines de
millions d'usagers à travers le monde utilisent désormais ces logiciels.
Parmi les plus populaires, on retrouve notamment le navigateur Firefox,
la suite bureautique OpenOffice et l'emblématique système d'exploitation
Linux, qui se développe d'ailleurs à partir de la licence GNU. Les
logiciels libres ont différents usages (en passant par la conception
Web, la gestion de contenu, les sytèmes d'exploitation, la
bureautique\ldots). Encore une fois, le logiciel libre est avant-tout
une philosophie, voire un mouvement de société. C'est une façon de
concevoir la communauté du logiciel, où le respect de la liberté de
l'utilisateur est un impératif éthique central
\textcolor{red}{(reformuler?)} (Williams et al., 2020:26). Si ce
mouvement fut d'abord initié par quelques militants dans les années
1980, c'est aujourd'hui un véritable phénomène sociétal: des milliers
d'entreprises, d'organisation à but non lucratif, d'institutions ou
encore de particuliers adoptent tour à tour ces logiciels, dont la
culture globale et les valeurs (entraide, collaboration, partage)
s'arriment avec le virage technologique de plusieurs entreprises à l'ère
du numérique \textcolor{red}{(retravailler, mais l'idée est là)}.
{[}blablabla{]}

Il faut garder en tête que logiciel libre ne rime pas nécessairement
avec gratuité. Bien que plusieurs logiciels libres soient
téléchargeables gratuitement \textcolor{red}{(donner des exemples)}, il
est aussi possible de (re)distribuer des logiciels libres payants
\textcolor{red}{(reformuler, pas clair)}. Par ailleurs, aucun logiciel
libre n'est réellement «gratuit» dans la mesure où son déploiment et son
utilisation nécessitent généralement différents coûts, dont les degrés
sont variables en fonction des compétences et de l'infrastructure dont
disposent les utilisateurs (coût d'apprentissage, coûts d'entretien,
etc.). Enfin, il est important de garder en tête les logiciels libres
possèdeux eux-aussi une licence - cette dernière est d'ailleurs garante
des libertés que confèrent les logiciels libres aux utilisateurs.

\hypertarget{logiciel-libre-et-open-source}{%
\subsection{\texorpdfstring{Logiciel libre et \emph{open
source}}{Logiciel libre et open source}}\label{logiciel-libre-et-open-source}}

``Les deux expressions décrivent à peu près la même catégorie de
logiciel, mais elles représentent des points de vue basés sur des
valeurs fondamentalement différentes. L'open source est une méthodologie
de développement ; le logiciel libre est un mouvement de société.''

\hypertarget{principaux-avantages-et-inconvuxe9nients}{%
\section{Principaux avantages et
inconvénients}\label{principaux-avantages-et-inconvuxe9nients}}

La disponibilité du code source et le mode de développement collaboratif
du logiciel libre facilitent également le transfert des connaissances et
ce, au-delà des frontières. Où qu'ils soient, les institutions, les
entreprises et les particuliers peuvent utiliser ces logiciels et les
adapter en fonction de leurs besoins respectifs. Par ailleurs, l'accès
libre et égal de tous les internautes à l'ensemble de ces connaissances
constitue un enjeu majeur pour la vitalité démocratique des sociétés à
l'ère du numérique, caractérisées par une surabondace d'information.

Les logiciels libres, parce qu'ils sont souvent moins coûteux (voire
téléchargeables gratuitement) et qu'ils démocratisent l'accès à
l'information, contribuent à réduire les disparités en termes
d'accessibilité aux nouvelles technologies.

Stallman - Lui-même issu du monde de la recherche scientifique. L'esprit
même du logiciel libre est très proche ; contribution à la culture
globale de partage, d'entraide, etc. que l'on peut retrouver dans le
domaine scientifique

\bookmarksetup{startatroot}

\hypertarget{r-ou-ne-pas-r}{%
\chapter{R ou ne pas R?}\label{r-ou-ne-pas-r}}

\begin{flushright}
William Poirier
\end{flushright}

À ce point du livre, vous avez été introduit aux problèmes et aux
opportunités qu'amène l'ère numérique. Vous avez d'ailleurs sans doute
déjà une idée de comment ou d'à propos de quel sujet vous pourriez
exploiter ce nouveau monde de possibilité. Les prochaines sections du
livre auront ainsi pour but de vous introduire aux outils qui
permettront la réalisation de vos projets, en commençant par l'outil
d'analyse de base -- le langage R.

\hypertarget{pourquoi-r}{%
\section{Pourquoi R?}\label{pourquoi-r}}

R est un langage de programmation \emph{OpenSource} développé par des
statisticiens pour des statisticiens dans les années 1990 (Tippmann
2015). C'est d'un élan d'amour propre et du désire d'honorer le langage
de programmation S que Ross Ihaka et Robert Gentleman nommeront leur
création, infirmant ainsi la légende selon laquelle les scientifiques
seraient mauvais pour nommer les choses. Ces derniers feront des choix
non orthodoxes lors de l'élaboration du langage, des choix qui font
aujourd'hui la popularité de R auprès d'un large pan de la communauté
académique. En effet, Morandat et al. (2012) rapportent que le langage a
été élaboré afin qu'il soit intuitif et qu'il permette aux nouveaux
utilisateurs de rapidement réaliser des analyses. Ils rapportent même
que dans plusieurs départements de statistiques, R est introduit en 2
semaines -- environ le temps que prend l'individu moyen pour oublier ses
résolutions du Nouvel An.

Toutefois, avant de débuter l'apprentissage d'un nouvel outil, il faut
être convaincu de sa pertinence, de son utilité. À quoi bon apprendre à
utiliser une perceuse alors que mon tournevis fonctionne parfaitement
bien ? C'est pourquoi ce chapitre a deux objectifs, d'abord, il s'agira
de vous convaincre de la pertinence de R suite à quoi il vous sera
introduit diverses utilisations possibles de R. Plus spécifiquement, la
section de réflexion théorique exposera les avantages et les
inconvénients de R et le comparera à ses principaux compétiteurs.
Ensuite, la réflexion méthodologique présentera brièvement la
programmation de base en R et en quoi l'OpenSource fait de R un outil si
puissant. Le chapitre se conclura avec quelques trucs et astuces qui
vous permettront de surmonter l'anxiété que peut causer l'apprentissage
d'un outil étant, pour plusieurs, quelque chose de véritablement
étranger à leur relation typique avec les ordinateurs.

\hypertarget{ruxe9flexion-thuxe9orique}{%
\section{Réflexion théorique}\label{ruxe9flexion-thuxe9orique}}

R a deux types de compétiteurs lorsqu'il est question d'analyses
statistiques -- les logiciels à licences comme SAS,STATA et SPSS, et les
langages \emph{OpenSource}, principalement Python et sa libraire Pandas.
Le chapitre précédant ayant déjà élaborer un cas exhaustif en faveur du
logiciel libre, il ne sera ici que rappelé les grandes lignes de
l'argument, à savoir que : 1) l'\emph{OpenSource} est gratuit
d'utilisation; 2) l'\emph{OpenSource} est développé de façon bottom-up,
ce qui lui procure une grande flexibilité; et 3) il permet aux
utilisateurs de créer leurs propres fonctions. À l'inverse, les
logiciels à licences sont coûteux, rigides et l'ajout de fonctionnalités
se fait par les développeurs internes à la compagnie ce qui rend le
processus plus lent et réduit l'éventail des possibilités. Ceci étant
dit, certains avanceront que le c'est justement ce processus interne
lent qui assure la validité et la fiabilité des analyses effectuées par
SAS, STATA ou SPSS. Or, dans son livre dédié aux utilisateurs de SPSS et
de SAS, Muenchen (2011) soulève le point que bien souvent, ce sont des
individus atomisés qui développent les nouvelles fonctionnalités de ces
langages et que le processus de révisions se fait ensuite par des
comités internes de testeurs. Il en va de même pour le développement des
\emph{Packages} R dans la mesure où ce dernier se vois tester et amender
par plusieurs programmeurs indépendants dans un processus itératif sur
GitHub ou sur d'autres plateformes similaires. De plus, bien des
nouvelles techniques statistiques sont développées pour R par des
professeurs qui publie d'abord leur travail dans des journaux
académiques revus par des pairs. Bien entendu, rien n'empêche un
étudiant gradué de publier ses propres \emph{packages}. C'est pourquoi
Muenchen (2011) recommande de visiter le site \emph{MACHIN} afin d'avoir
une idée de la validité et de la fiabilité du \emph{package} en
question. Enfin, le fait que SAS et SPSS permettent à leur utilisateur
d'intégrer des routines R à leur programme est un indicateur fort ne
serait-ce que de l'utilité de R (Muenchen 2011).

R n'est cependant pas qu'un outil statistique, il s'agit également d'un
outil de programmation puissant. Ceci fait en sorte que le coût d'entrer
de R est plus important que celui des logiciels comme SAS, STATA et SPSS
puisqu'il impose l'apprentissage d'une syntaxe et d'un jargon
particulier. Alors, pourquoi apprendre à programmer en plus d'apprendre
à réaliser des analyses statistiques? Après tout, faire des statistique
c'est déjà beaucoup! D'abord, apprendre à programmer permet de
développer la résolution de problème et la logique, deux compétences aux
cœur de la recherche scientifique. Programmer est au cerveau ce que
courir est au coeur, il s'agit d'un exercice difficile au départ mais
dont les résultats bénéfique se font sentir rapidement. Apprendre à
programmer permettra également de mieux comprendre la façon dont son
propre ordinateur fonctionne. Contrairement au mythe urbain qui veux que
l'humain n'utilise que 10\% de son cerveau, la plupart des individus ne
font qu'utiliser une infime partie du potentiel de leur ordinateur. Par
exemple, l'ordinateur ayant permis aux américains d'aller sur la lune
lors de la mission Apollo-11, le \emph{Apollo Guidance Computer (AGC)},
avait 4 096 octets (bytes) de mémoire RAM\footnote{La RAM (\emph{Random
  Access Memory}) ou ``mémoire vive'' est l'espace utilisée par
  l'ordinateur pour enregistrer l'information directement nécessaire
  pour les opérations en cours d'exécution. Elle s'oppose à la ROM
  (\emph{Read-Only Memory}) ou ``mémoire morte'' qui contient toute
  l'information enregistré de façon permanante dans l'ordinateur.}
{[}\emph{SOURCE FIABLE}{]}. L'ordinateur sur lequel sont écrits ces
lignes a 8GB de RAM, soit approximativement 2 millions de fois plus que
celle de l'AGC. Enfin, l'argument le plus probant sur la nécessité
d'apprendre à programmer est celui du marché de l'emplois. Au Canada, il
est prévu une pénurie de main d'oeuvre pour les emplois requérant de
aptitude en statistique et en programmation comme celui de scientifique
de données ou d'analyste (Employment and Social Development Canada
2023). Un rapport de PWC indique même que les employeurs devrons
s'attendre à ce battre pour engager des individus compétent dans les
deux domaines. Apprendre à programmer devient alors, non seulement, une
façon d'améliorer votre résonnement scientifique, mais également une
façon de vous démarquer sur le marché de l'emplois.

\begin{itemize}
\tightlist
\item
  Avantages :

  \begin{itemize}
  \tightlist
  \item
    Gratis!!
  \item
    Plus facile d'accès que d'autres langages de programmation
  \item
    Arguments technos :

    \begin{itemize}
    \tightlist
    \item
      Fait pour les stats
    \item
      Fait tout ce que SPSS et Stata font sans le carcan du logiciel
      privé
    \end{itemize}
  \item
    Arguments d'autorités \textless- popularité du langage
  \item
    Ouvre vers un monde de possibilités
  \item
    Développement d'une expertise recherchée
  \end{itemize}
\item
  Inconvénients :

  \begin{itemize}
  \tightlist
  \item
    Courbe d'apprentissage rude pour certains
  \item
    Développement anarchique
  \item
    Risque de se perdre dans les profondeurs pleines de microbes de CRAN
  \end{itemize}
\item
  Propriété et utilisation de R
\item
  Comparaison avec d'autres langages

  \begin{itemize}
  \tightlist
  \item
    Python, SPSS, Stata?
  \end{itemize}
\end{itemize}

\hypertarget{ruxe9flexion-muxe9thodologique}{%
\section{Réflexion
méthodologique}\label{ruxe9flexion-muxe9thodologique}}

\begin{itemize}
\tightlist
\item
  Base R

  \begin{itemize}
  \tightlist
  \item
    Stable mais parfois bof
  \item
    Manipulation de données
  \item
    Fonctions et Boucles
  \end{itemize}
\item
  La puissance de l'OpenSource

  \begin{itemize}
  \tightlist
  \item
    Peut être instable, mais souvent plus intéressant que les options en
    Base R
  \item
    TidyVerse

    \begin{itemize}
    \tightlist
    \item
      Manipulation avec Dplyr
    \item
      All hail Hadley!
    \end{itemize}
  \item
    Analyse textuelle avec ???
  \item
    Shiny
  \end{itemize}
\end{itemize}

\hypertarget{trucs-et-astuces}{%
\section{Trucs et astuces}\label{trucs-et-astuces}}

\begin{itemize}
\tightlist
\item
  10 choses à garder en tête lorsque l'on apprend R :

  \begin{enumerate}
  \def\labelenumi{\arabic{enumi}.}
  \tightlist
  \item
    Vous n'allez pas briser votre ordinateur.
  \item
    C'est en ``gossant'' que l'on apprend! Essayez des trucs,
    expérimentez -- souvenez-vous de 1.
  \item
    Contrairement à la vraie vie, il y a toujours le crtl-z pour vous
    sauver.
  \item
    Ayez de l'empathie pour vos futurs lecteurs, ou du moins pour le
    futur vous -- COMMENTEZ VOTRE CODE!
  \item
    Même Wozniak était mauvais au début.
  \item
    Pensez à votre santé -- levez-vous de votre chaise aux 30 minutes.
  \item
    S'il y a un bogue, et il y en aura, \emph{Google} est votre ami.
  \item
    Souvent c'est une question de type de variable -- caractère,
    numérique, facteurs, etc.
  \item
    Si vous faites beaucoup de copier-coller de code, il y a surement
    une façon de l'automatiser.
  \item
    Sérieusement, faites le 4!
  \end{enumerate}
\end{itemize}

\bookmarksetup{startatroot}

\hypertarget{les-environnements-de-duxe9veloppement-intuxe9gruxe9}{%
\chapter{Les environnements de développement
intégré}\label{les-environnements-de-duxe9veloppement-intuxe9gruxe9}}

\hypertarget{ouxf9-coder-en-r}{%
\section{Où coder en R ?}\label{ouxf9-coder-en-r}}

Un environnement de développement intégré (IDE), permet aux programmeurs
de consolider les différents aspects de l'écriture d'un programme
informatique. Ils permettent de réaliser toutes les activités courantes
d'un programmeur -- l'édition du code, la construction des exécutables
et le débogage -- au même endroit. Les environnements de développement
intégrés sont conçus pour maximiser la productivité du programmeur. Ils
fournissent de nombreuses fonctionnalités -- notamment la coloration
syntaxique et le contrôle de version -- pour créer, modifier et compiler
du code.

Certains environnements de développement intégré sont dédiés à un
langage de programmation spécifique. Par conséquent, ils contiennent des
fonctionnalités qui sont plus compatibles avec les paradigmes de
programmation du langage auquel is sont associés. Cependant, il existe
de nombreux environnements de développement intégré multilingues.

R est un des langages de statistiques et d'exploration de données les
plus populaires en sciences sociales et il est open-source. Par
conséquent, il est logique de choisir un environnement de programmation
open-source. R est pris en charge par de nombreux environnements de
programmation. Plusieurs ont été spécialement conçus pour la
programmation en R -- le plus notable étant RStudio -- tandis que
d'autres sont des environnement de programmation universels -- tel que
Visual Studio -- et prennent en charge R via des plugins. Il est
également possible de coder en R à partir d'une interface en ligne de
commande. Une telle méthode permet la communication entre l'utilisateur
et son ordinateur. Cette communication s'effectue en mode texte :
l'utilisateur tape une « ligne de commande » -- c'est-à-dire du texte
dans le \emph{terminal} -- pour demander son ordinateur d'effectuer une
opération précise, par exemple rouler un fichier de code R.

Le présent chapitre présente RStudio, ses avantages et inconvénients
ainsi que des exemples de ses fonctionnalités de RStudio et des conseils
sur comment l'utiliser et le personnaliser.

\hypertarget{pourquoi-rstudio}{%
\section{Pourquoi RStudio ?}\label{pourquoi-rstudio}}

\hypertarget{quest-ce-que-rstudio}{%
\subsection{Qu'est-ce que RStudio ?}\label{quest-ce-que-rstudio}}

\textless\textless\textless\textless\textless\textless\textless{} HEAD
\#\# Pourquoi RStudio ?

\hypertarget{quest-ce-que-rstudio-1}{%
\subsection{Qu'est-ce que RStudio ?}\label{quest-ce-que-rstudio-1}}

Comme plusieurs autres langages de programmation, R est développé grâce
à des fonctions écrites par ses usagers. Un IDE, comme RStudio, est
conçu pour faciliter ce travail (Verzani, 2011). RStudio est un projet
open source destiné à combiner les différentes composantes du langage de
programmation R en un seul outil (Allaire, 2011). Il est conçu pour
faciliter la courbe d'apprentissage des nouveaux utilisateurs. RStudio
fonctionne sur toutes les systèmes d'exploitation, y compris Windows,
Mac OS et Linux. En plus de l'application de bureau, RStudio peut être
déployé en tant que serveur pour permettre l'accès Web aux sessions R
s'exécutant sur des systèmes distants (Allaire, 2011).

\emph{Figure of RStudio with some code, a plot in the bottom right
corner and some data in the top right corner}

RStudio facilite l'utilisation du langage de programmation R en offrant
de nombreux outils permettant à son utlisateur d'aisément réaliser ses
tâches. Parmi les plus utiles, on retrouve notamment une fenêtre d'aide,
de la documentation sur les différents packages R, un navigateur
d'espace de travail, une visionneuse de données et une prise en charge
de la coloration syntaxique (Horton, Kleinman, 2015). De plus, RStudio
permet de coder dans plusieurs langages et supportent un grande quantité
de formats. Il fournit également un support pour plusieurs projets ainsi
qu'une interface pour utiliser des systèmes de contrôle des versions
tels que GitHub (Horton, Kleinman, 2015).

\hypertarget{avantages-et-inconvuxe9nients-de-rstudio}{%
\subsection{Avantages et inconvénients de
RStudio}\label{avantages-et-inconvuxe9nients-de-rstudio}}

RStudio a plusieurs avantages. L'utilisation de l'IDE est facile à
apprendre pour les débutants. Les principaux éléments d'un IDE sont
intégrés dans une disposition à quatre volets (Verzani, 2011). Cette
disposition comprend une console, un éditeur de code source à onglets
pour organiser les fichiers d'un projet, un espace pour l'environnement
de travail et un quatrième volet où il est possible d'afficher des
graphiques ou de la documentation sur différents packages. De plus, on y
retrouve la possibilité de créer plusieurs espaces de travail -- appelés
projets -- qui facilitent l'organisation de différents workflows.

Un autre aspect de RStudio que de nombreux programmeurs apprécient est
le fait qu'il peut être utilisé via un navigateur Web pour un accès à
distance (Verzani, 2011). De plus, l'IDE offre de nombreux outils
pratiques et faciles à utiliser pour gérer les packages, l'espace de
travail et les fichiers. RStudio supporte plusieurs langages de
programmation ainsi que différents langages de balisage. Finalement, de
nouvelles fonctionnalités sont souvent ajoutées pour satisfaire aux
besoins de la communauté scientifique et le logiciel est régulièrement
mis à jour.

Inconvénients : - Peu de configuration, tu peux pas changer les
raccourcis, etc - Très limité dans le setup des différents panneaux, tu
peux pas voir 2 fichiers en même temps Tu peux pas visualiser 2 fichiers
1 à côté de l'autre, une feature de base dans n'importe quel ide Tu peux
pas configurer tes raccourcis clavier, aussi une feature de base Mettons
que je veux que ctrl + D copie la ligne, comme dans visual studio, je
peux pas - Plus lent que d'autres alternatives pour certaines opérations

======= Comme plusieurs autres langages de programmation, R est
développé grâce à des fonctions écrites par ses usagers. Un IDE, comme
RStudio, est conçu pour faciliter ce travail (Verzani, 2011). RStudio
est un projet open source destiné à combiner les différentes composantes
du langage de programmation R en un seul outil (Allaire, 2011). Il est
conçu pour faciliter la courbe d'apprentissage des nouveaux
utilisateurs. RStudio fonctionne sur toutes les systèmes d'exploitation,
y compris Windows, Mac OS et Linux. En plus de l'application de bureau,
RStudio peut être déployé en tant que serveur pour permettre l'accès Web
aux sessions R s'exécutant sur des systèmes distants (Allaire, 2011).

\emph{Figure of RStudio with some code, a plot in the bottom right
corner and some data in the top right corner}

RStudio facilite l'utilisation du langage de programmation R en offrant
de nombreux outils permettant à son utlisateur d'aisément réaliser ses
tâches. Parmi les plus utiles, on retrouve notamment une fenêtre d'aide,
de la documentation sur les différents packages R, un navigateur
d'espace de travail, une visionneuse de données et une prise en charge
de la coloration syntaxique (Horton, Kleinman, 2015). De plus, RStudio
permet de coder dans plusieurs langages et supportent un grande quantité
de formats. Il fournit également un support pour plusieurs projets ainsi
qu'une interface pour utiliser des systèmes de contrôle des versions
tels que GitHub (Horton, Kleinman, 2015).
\textgreater\textgreater\textgreater\textgreater\textgreater\textgreater\textgreater{}
0767363f696a8e4a9fac38bd0de1c9327b812608

\hypertarget{avantages-et-inconvuxe9nients-de-rstudio-1}{%
\subsection{Avantages et inconvénients de
RStudio}\label{avantages-et-inconvuxe9nients-de-rstudio-1}}

\textless\textless\textless\textless\textless\textless\textless{} HEAD -
Exemples de fonctionnalités + Files, Plots, Packages, Help, and Viewer
Pane Layout of the Components The RStudio interface consists of several
main components sitting below a top-level toolbar and menu bar. Although
this placement can be customized, the default layout utilizes four main
panes in the following positions:

In the upper left is a Source browser pane for editing files (see Source
Code Editor) or viewing some data sets. In Figure 1-3 this is not
visible, as that session had no files open.

In the lower left is a Console for interacting with an R process (see
Chapter 3).

In the upper right are tabs for a Workspace browser (see the section
Workspace Browser) and a History browser (see the section Command
History).

In the lower right are tabbed panes for interacting with the Files (The
File Browser), Plots (Graphics in RStudio), Packages (Package
Maintenance), and Help system components (The Help Page Viewer). If the
facilities are present, an additional tab for version control (Version
Control with RStudio) is presented.

The Console pane is somewhat privileged: it is always visible, and it
has a title bar. For the other components, their tab serves as a title
bar. These panes have page-specific toolbars (perhaps more than
one)---which in the case of the Source pane are also context-specific.

The user may change the default dimensions for each of the panes, as
follows. There is an adjustable divider appearing in the middle of the
interface between the left and right sides that allows the user to
adjust the horizontal allocation of space. Furthermore, each side then
has another divider to adjust the vertical space between its two panes.
As well, the title bar of each pane has icons to shade a component,
maximize a component vertically, or share the space (Verzani, 2011).
\textbf{Also check Nierhoff et Hillebrand 2015}

\begin{itemize}
\tightlist
\item
  Comment personnaliser / se familiariser avec RStudio

  \begin{itemize}
  \item
    Changement de couleurs pour le fond
  \item
    \bookmarksetup{startatroot}

    \hypertarget{section-aide}{%
    \chapter{Section aide}\label{section-aide}}

    RStudio a plusieurs avantages. L'utilisation de l'IDE est facile à
    apprendre pour les débutants. Les principaux éléments d'un IDE sont
    intégrés dans une disposition à quatre volets (Verzani, 2011). Cette
    disposition comprend une console, un éditeur de code source à
    onglets pour organiser les fichiers d'un projet, un espace pour
    l'environnement de travail et un quatrième volet où il est possible
    d'afficher des graphiques ou de la documentation sur différents
    packages. De plus, on y retrouve la possibilité de créer plusieurs
    espaces de travail -- appelés projets -- qui facilitent
    l'organisation de différents workflows.
  \end{itemize}
\end{itemize}

Un autre aspect de RStudio que de nombreux programmeurs apprécient est
le fait qu'il peut être utilisé via un navigateur Web pour un accès à
distance (Verzani, 2011). De plus, l'IDE offre de nombreux outils
pratiques et faciles à utiliser pour gérer les packages, l'espace de
travail et les fichiers. RStudio supporte plusieurs langages de
programmation ainsi que différents langages de balisage\footnote{À cet
  effet, voir le chapitre 6 du présent ouvrage.}. Finalement, de
nouvelles fonctionnalités sont souvent ajoutées pour satisfaire aux
besoins de la communauté scientifique et le logiciel est régulièrement
mis à jour.

L'IDE comporte toutefois certains inconvénients. En effet, il n'est pas
possible de configurer des raccourcis claviers personnalisés, une
possibilité que plusieurs autres IDE offrent. De plus, les possibilités
de disposition des différents panneaux sont très limitées et la
malléabilité de l'espace de travail est restreinte. Finalement, pour
certaines opérations, RStudio peut être plus lent que d'autres
alternatives.

\hypertarget{comment-utiliser-rstudio}{%
\section{Comment utiliser RStudio ?}\label{comment-utiliser-rstudio}}

La première étape pour commencer à utiliser RStudio est de
l'installer\footnote{À cet effet, voir le chapitre 9 du présent ouvrage.}.
Une fois que cela est fait, ouvrez RStudio. La fenêtre qui apparait
devrait ressembler à l'image ci-dessus. La couleur de l'arrière-plan et
celle de la police, la taille des cadrans ainsi que de nombreux autres
éléments peuvent être changés. La dernière section de ce chapitre aidera
le lecteur à personnaliser son IDE.

\textbf{Image de RStudio sans rien, settings de base}.

Bien que de nombreux éléments puissent être personnalisés, la
disposition par défaut de RStudio est composée de quatre volets
principaux (Verzani, 2011). Dans le coin supérieur gauche se trouve le
quadran principal. C'est dans celui-ci que l'utilisateur passera la plus
grande partie de son temps. On y modifie des fichiers de différents
formats et il est possible d'y afficher des bases de données. Dans le
coin inférieur gauche se trouve la console et le terminal. Dans cette
première, on peut interagir avec R de la même manière que dans le cadran
principal, mais le code ne sera pas enregistré. Le terminal, pour sa
part, est le point d'accès de communication entre un usager et son
ordinateur. Bien que les différents systèmes d'exploitation viennent
avec un terminal déjà intégré, il est aussi possible d'y accéder a
partir de RStudio.
\textgreater\textgreater\textgreater\textgreater\textgreater\textgreater\textgreater{}
0767363f696a8e4a9fac38bd0de1c9327b812608

\textbf{Image de RStudio qui montre les quatre cadrans. Idéalement avec
un projet en cours et les différents cadrans utilisés. Settings
personalisés?}.

\textless\textless\textless\textless\textless\textless\textless{} HEAD
One can easily switch between components using the mouse. As well, the
View menu has subitems for this task. For power users, the keyboard
shortcuts listed in Table 1-2 are useful. (A full list of keyboard
shortcuts is available through the Help \textgreater{} Keyboard
Shortcuts menu item.) ======= On retrouve dans le coin supérieur droit
l'espace de travail. Ce cadran contient trois éléments :
l'\emph{environnement global, l'historique et les connections}.
L'\emph{environnement global} est l'endroit où l'utilisateur peut voir
les bases de données, les fonctions et les différents autres objets R
qui sont actifs. Il peut cliquer sur les divers éléments actifs pour les
consulter. L'onglet \emph{historique} permet à l'utilisateur de
consulter les derniers morceaux de code R qu'il a roulé ainsi que les
dernières commandes écrites dans la console. L'onglet
\emph{connections}, pour sa part, permet de connecter son IDE à une
variété de sources de données et d'explorer les objets et les données
qui la compose. Il est conçu pour fonctionner avec une variété d'autres
outils pour travailler avec des bases de données en R dans RStudio.

Le cadran dans le coin inférieur droit, pour sa part, contient plusieurs
outils très utiles pour les usagers de RStudio. L'onglet \emph{Files}
permet à l'utilisateur de naviguer dans les fichiers que contient son
ordinateur sans avoir à sortir de RStudio. L'onglet \emph{Plots} permet
de visualiser les graphiques générer à partir de R, que ce soit en
utilisant \emph{ggplot2, lattice ou base R}\footnote{Pour en apprendre
  davantage sur la visualisation graphique en R, consulter le chapitre 7
  du présent ouvrage.}. L'ongliet \emph{Packages} permet de consulter
les packages installés précédemment par l'utilisteur en plus de pouvoir
en consulter la documentation. C'est aussi un des différents endroits à
partir d'où il est possible d'installer des packages avec RStudio.
L'onglet \emph{Help} permet à l'utilisateur de chercher et de consulter
de la documentation sur de nombreux sujets, notamment sur les
différentes fonctions en R ainsi que sur les packages. Pour sa part,
l'onglet \emph{Viewer} permet la visualisation de contenu web local.

L'utilisateur peut modifier les dimensions par défaut pour chacun des
quatre cadrans principaux. En cliquant sur la division des sections, il
est possible d'ajuster l'allocation horizontale de l'espace. De plus,
chaque côté dispose d'un autre séparateur pour ajuster l'espace
vertical. Qui plus est, la barre de titre de chaque cadran comporte des
icônes pour ombrer un composant, maximiser un cadran verticalement ou
modifier la taille des l'espace de travail (Verzani, 2011; Nierhoff et
Hillebrand, 2015).

\hypertarget{personnaliser-son-rstudio}{%
\section{Personnaliser son RStudio}\label{personnaliser-son-rstudio}}

One can easily switch between components using the mouse. As well, the
View menu has subitems for this task. For power users, the keyboard
shortcuts listed in Table 1-2 are useful. (A full list of keyboard
shortcuts is available through the Help \textgreater{} Keyboard
Shortcuts menu item.)

\begin{itemize}
\tightlist
\item
  Comment personnaliser / se familiariser avec RStudio

  \begin{itemize}
  \tightlist
  \item
    Changement de couleurs pour le fond
  \item
    Section aide
    \textgreater\textgreater\textgreater\textgreater\textgreater\textgreater\textgreater{}
    0767363f696a8e4a9fac38bd0de1c9327b812608
  \end{itemize}
\end{itemize}

\bookmarksetup{startatroot}

\hypertarget{baliser-les-sciences-sociales-langages-et-pratiques}{%
\chapter{Baliser les sciences sociales~: langages et
pratiques}\label{baliser-les-sciences-sociales-langages-et-pratiques}}

\hypertarget{question}{%
\section{Question}\label{question}}

Lorsque vous lisez une page Web, un article scientifique ou un
curriculum vitæ professionnel, vous vous doutez peut-être que le texte
n'est pas toujours produit à l'aide d'un simple logiciel de traitement
de texte comme Microsoft Word, Apple Pages ou LibreOffice Writer. La
mise en page réglée au millimètre près, la qualité des figures et
graphiques, le style des références, la présence d'éléments interactifs
et la cohérence hiérarchique du texte sont difficiles à reproduire à
l'aide d'un logiciel de traitement de texte régulier, entre autres.
L'insertion de tableaux de régression, de figures et d'extraits de code
de haute qualité graphique ainsi que leur personnalisation nécessitent
une interface particulière.

Pour ces raisons et plusieurs autres, les chercheurs en sciences
sociales font souvent appel aux langages de balisage, ou \emph{markup
languages}. Ceux-ci permettent de produire des documents et pages Web
sans les limitations des logiciels de traitement de texte. Le présent
livre, par exemple, est écrit à l'aide du langage de balisage Markdown.
D'entrée de jeu, vous vous demandez peut-être quelle est l'utilité
d'apprendre ces langages alors que les logiciels de traitement de texte
sont nombreux, simples d'approche et en amélioration constante. Ce
chapitre tentera donc de répondre aux questions suivantes~: «~Pourquoi
apprendre à utiliser des langages de balisage? Dans quels contextes
sont-ils plus utiles que les logiciels de traitement de texte? Comment
les utiliser?~» L'accent sera mis sur R Markdown, \LaTeX, BibTeX et
HTML.

Le premier langage de balisage, le Generalized Markup Language (GML), a
été inventé en 1969 par les chercheurs Charles F. Goldfarb, Ed Mosher et
Ray Lorie pour la compagnie IBM. Goldfarb et ses collègues devaient
intégrer trois applications créées avec des langages différents et avec
une logique différente pour les besoins d'un bureau de droit. Même après
avoir créé un programme qui permettait aux trois applications
d'interagir, ces langages demeuraient différents et avaient chacun leur
propre fonctionnement. Le développement de GML a permis de résoudre ce
problème en standardisant et en structurant le langage~: les mêmes
commandes étaient utilisées pour accomplir les mêmes tâches dans chaque
programme (Goldfarb 1996). GML a été amélioré durant les décennies
suivantes et a été suivi par d'autres langages de balisage, dont
\LaTeX~(1984), HTML (1993), XML (1998) et Markdown (2004).

Un langage de balisage constitue un ensemble de commandes qui peuvent
être entremêlées à du texte afin de produire une action informatique.
Chaque langage contient son ensemble de commandes cohérentes et
complémentaires. De manière plus formelle, ces commandes sont nommées
\emph{balises} (\emph{tags} en anglais) et inscrites par le chercheur
lui-même au travers du texte. Les balises constituent une manière de
communiquer avec le logiciel que vous utilisez dans un langage qu'il
peut comprendre, par exemple pour lui indiquer que vous désirez qu'une
section du texte soit écrite en caractères gras, en italique, à double
interligne ou encore que vous souhaitez positionner une image d'une
certaine manière au travers du texte. Cette interaction est rendue
possible par la standardisation des langages de balisage~: chaque balise
correspond à une action précise, peu importe le logiciel utilisé, la
langue dans laquelle le texte est rédigé, le type d'ordinateur utilisé,
etc. Dans votre document source, les balises sont entremêlées au contenu
de votre document, puis au moment de compiler ce dernier, les balises
disparaissent, produisent les actions informatisées qu'elles commandent
et ne laissent comme document final que son contenu mis en page tel que
vous l'avez défini via les balises utilisées.

Plusieurs langages de balisage existent et permettent d'effectuer
différentes tâches. Le plus répandu est le langage HTML, qui permet de
formater des sites web. Le langage XML, lui aussi très utilisé, permet
de structurer de larges volumes de données. \LaTeX~permet pour sa part
de formater du texte et de créer des documents en format PDF. Markdown
permet également de créer des documents de format PDF, mais aussi HTML
et DOCX. Depuis 2014, le \emph{package} R Markdown permet d'ajouter des
extraits de code R à un fichier en langage Markdown. \LaTeX~et Markdown
permettent aussi d'intégrer les références bibliographiques du système
de traitement de références BibTeX, créé en 1985.

Les balises constituent une manière de donner manuellement des commandes
au logiciel que vous utilisez. Par exemple, si vous utilisez Microsoft
Word, vous avez accès à une panoplie de boutons qui vous permettent de
formater votre texte. Les balises exercent les mêmes fonctions, mais de
manière manuelle. Lorsque vous appuyez sur un bouton dans Word, celui-ci
ajoute des balises au travers de votre texte, mais rend celles-ci
invisibles dans l'interface que vous utilisez. Cela permet d'avoir un
texte élégant et facile à lire, mais comporte aussi plusieurs
inconvénients. Le principal inconvénient est que vous êtes condamné à
avoir un pouvoir limité sur le formatage de votre texte. En effet, si
les boutons à votre disposition ne vous permettent pas de réaliser une
opération, celle-ci sera éternellement impossible à réaliser pour vous.
A contrario, les langages de balisage permettent un contrôle presque
infini sur les opérations que vous souhaitez réaliser. Incidemment, dans
la mesure où vous utilisez le langage approprié pour la tâche que vous
souhaitez accomplir, vous devriez être capable de donner exactement la
commande nécessaire à votre logiciel. Les langages de balisage, bien
qu'ils aient un coût d'apprentissage qui peut s'avérer important et
qu'ils soient moins élégants qu'un simple document Word, vous offrent
une plus grande flexibilité.

Afin d'utiliser un langage de balisage, il est impératif que le logiciel
que vous utilisez puisse prendre en compte ce langage. Un logiciel
permet rarement d'utiliser n'importe quel langage. Il est aussi
impératif de bien utiliser le langage de balisage. En effet, comme pour
les langages de programmation, les langages de balisage ne peuvent pas
déduire ce que vous souhaitez leur faire comprendre. Si vous souhaitez
mettre du texte en gras, vous devez utiliser les bonnes balises. La
moindre erreur est fatale, puisqu'une erreur dans la balise que vous
utilisez produira un message d'erreur, le logiciel ne réussissant pas à
associer votre balise mal inscrite à une action informatisée.
Conséquemment, il est impératif de bien vérifier les balises utilisées
afin d'éviter toute erreur qui empêcherait votre document d'être
compilé, c'est-à-dire d'être traduit dans son format final.\footnote{Les
  logiciels permettent plus ou moins efficacement d'identifier les
  balises problématiques. Certains ne produisent qu'un message d'erreur
  sans donner d'indication sur la source du problème, alors que d'autres
  ciblent très spécifiquement la ligne de syntaxe où se situe la balise
  problématique.} Chaque caractère dans une balise est important et il y
a rarement plus d'une seule manière de commander une action. Le
positionnement des balises est lui aussi critique~: il délimite la
portion de texte à laquelle doit être appliquée l'action commandée par
la balise.

Il est important de distinguer les langages de balisage des langages de
programmation. En effet, ceux-ci sont similaires à certains égards, mais
ont des vocations différentes. Les deux s'appuient sur un langage
informatisé, mais les langages et leurs objectifs diffèrent. Un langage
de programmation définit des processus informatisés alors qu'un langage
de balisage permet d'encoder du contenu de manière à ce que celui-ci
soit lisible tant pour l'humain que pour son ordinateur.

Dans le contexte de la recherche en sciences sociales, la programmation
est généralement utilisée afin de récolter, d'analyser et de présenter
visuellement des données. Une fois cartes, tableaux et graphiques
produits, ceux-ci peuvent être enregistrés -- par exemple en format PDF
ou PNG -- et inclus au sein d'un document qui sera formaté en utilisant
un langage de balisage. De manière simple, le langage de programmation
contribue à l'analyse alors que le langage de balisage est
essentiellement utile afin de présenter les travaux de recherche, que ce
soit dans un document écrit ou sur un site web. C'est principalement de
cette manière que sont utilisés les langages de programmation et de
balisage dans le cadre de la recherche en sciences sociales.

\hypertarget{ruxe9flexion-thuxe9orique-1}{%
\section{Réflexion théorique}\label{ruxe9flexion-thuxe9orique-1}}

La plupart des langages de balisage permettent de remplir l'une des deux
fonctions suivantes, qui sont particulièrement importantes dans le
contexte de la recherche en sciences sociales~: produire des documents
écrits et gérer des pages Web. Dans les deux cas, cependant, certains
sites Web et applications permettent également de remplir ces fonctions,
mais avec des limites importantes.

Pour l'écriture de documents très simples comme une liste d'épicerie ou
des notes rapides pendant une conférence, les logiciels de traitement de
texte sont tout-à-fait convenables~: ils sont simples et rapides à
utiliser, un formatage professionnel du document n'est pas de mise.
Utiliser un langage de balisage pour des tâches de base n'est en effet
pas nécessaire. Par contre, plus la complexité d'un document augmente,
plus il devient difficile d'obtenir un résultat satisfaisant en
utilisant un logiciel de traitement de texte tel que Word, Pages ou
Writer. A contrario, \LaTeX~permet de produire des documents de tous les
niveaux de complexité, tel que démontré sur la Figure
\ref{latex-vs-word}. Plus généralement, utiliser un langage de balisage
comme \LaTeX~ou Markdown\footnote{Les avantages et désavantages de
  Markdown cités dans la prochaine section s'appliquent eux aussi à R
  Markdown. Les avantages ou inconvénients ne s'appliquant qu'à R
  Markdown et non à Markdown tout court sont présentés comme tels.}
comporte plusieurs avantages par rapport aux logiciels de traitement de
texte traditionnels. Ces avantages peuvent se résumer en quatre
concepts~: automatisation, personnalisation, flexibilité et qualité
graphique.

\begin{figure}
\begin{center}
\caption{Utilité relative de Word et \LaTeX\ selon la complexité et la taille du document \label{latex-vs-word}}
\includegraphics[width=0.75\textwidth]{../_SharedFolder_livre-ssn/Graphiques/Chapitre6/word-vs-latex.png}
\end{center}
\footnotesize{Source~: Yannick Dufresne (2015).}
\end{figure}

\emph{et Markdown sur la figure?}

Premièrement, \LaTeX~et Markdown permettent d'intégrer une bibliographie
\emph{automatique} et professionnelle en utilisant BibTeX. Cette
bibliographie peut être adaptée très facilement en différents styles
bibliographiques reconnus ou en un style bibliographique personnalisé.
Avec BibTeX, plus besoin de vérifier si le titre de l'article est
toujours en italique, si le numéro de volume est toujours entre
parenthèses ou si le nom de famille des deuxièmes auteurs est toujours
avant ou après le prénom puisque tout ceci est fait de manière
\emph{automatique}. BibTeX comprend également les différences entre les
types de sources (articles scientifiques, livres, sites Internet, etc.)
et ajuste leur présentation en conséquence. De plus, si une des sources
que vous citez n'est pas inclue dans la bibliographie, une erreur
s'affiche, vous permettant d'identifier le problème plutôt que de vous
retrouver avec une référence manquante. À l'inverse, si une source est
retirée du texte, elle disparait \emph{automatiquement} de la
bibliographie dans le document final mais demeure présente dans le
fichier où se trouvent les références bibliographiques. Cela évite les
aller-retour pour vérifier que chaque source de la bibliographie se
trouve au moins une fois dans le texte et que chaque source dans le
texte est citée en bibliographie. Grâce aux balises, en cliquant sur les
références incluses dans le document, celui-ci change de page pour se
retrouver automatiquement à l'entrée bibliographique associée. Les
références BibTeX pour articles scientifiques peuvent être
copiées-collées à partir de Google Scholar. BibTeX rend donc extrêmement
simple et efficace l'utilisation des références bibliographiques grâce à
sa capacité à \emph{personaliser} et \emph{automatiser} leur
présentation.

L'intégration de figures et tableaux dans le texte est aussi rendue très
simple et professionnelle grâce à \LaTeX~et Markdown. La taille de la
figure ou du tableau, son positionnement et son intégration par rapport
au texte environnant peuvent être réglés de telle sorte que l'ajout de
texte avant ou après la figure ou le tableau ne produira pas des
résultats inattendus. Au contraire, en définissant des paramètres pour
l'ensemble du texte, le chercheur pourra \emph{personaliser} entièrement
la présentation des figures et tableaux. De plus, la qualité des figures
et tableaux ne diminue pas lors de leur intégration~: les figures
restent aussi belles qu'elles l'étaient originalement, ce qui n'est pas
toujours le cas avec certains logiciels de traitement de texte. Les
numéros des figures et tableaux sont aussi mis-à-jour
\emph{automatiquement}, ce qui veut dire que vous n'aurez jamais à vous
préoccuper de modifier leur numéro lorsque vous rajoutez une figure ou
un tableau dans le texte. Grâce aux balises, en cliquant sur le numéro
associé à la figure ou au tableau dans le texte, le document se retrouve
automatiquement à l'endroit où se trouve le graphique ou tableau.

Markdown et \LaTeX permettent aussi la gestion \emph{automatisée} de la
table des matières, et les références aux pages appropriées se mettent à
jour en continu. La table des matières prend en compte l'architecture du
texte choisie manuellement par le chercheur, qui est définie par des
balises définissant différents niveaux hiérarchiques de sections,
sous-sections ou chapitres. Des manières \emph{automatiques} de
référencer les figures et les tableaux dans des sections distinctes de
la table des matières sont également offertes, encore une fois
\emph{personnalisables} au goût du chercheur.

Bien que la mise en page de documents produits via Markdown et
\LaTeX~puisse être définie entièrement manuellement par un utilisateur
expérimenté, les débutants apprécieront les nombreux gabarits
(\emph{templates}) disponibles en libre permettent de gérer
\emph{automatiquement} la mise en page de manière clé-en-main. Ceux-ci
permettent de rendre l'apparence d'un document plus esthétique et
uniforme et peuvent être utilisés tels quels ou peuvent servir de point
de départ pour un chercheur souhaitant y apporter certaines
modifications sans toutefois partir d'une feuille blanche. La majorité
des utilisateurs, même les plus expérimentés, utilisent ces gabarits
commme base lorsqu'ils rédigent un document. Ceux-ci constituent une
mine d'or puisqu'ils rendent accessible le code Markdown et \LaTeX~ayant
servi à la conception du gabarit, permettant au chercheur de comprendre
comment est obtenu le résultat que lui offre le gabarit. Incidemment, le
chercheur peut identifier les sections de code produisant certains
éléments de mise en page (ex: positionnement des numéros de page,
positionnement du nom des auteurs, etc.) et les modifier ou s'en
inspirer afin de modifier d'autres gabarits. L'utilisation de ces
gabarits peut s'avérer complexe pour les non-initiés, mais il s'agit
d'une complexité qui s'avère ultimement extrêmement productive
puisqu'elle permet au chercher de devenir autonome et d'ajuster les
gabarits à sa convenance afin de produire exactement le résultat désiré
en terme de mise-en-page. La liste des gabarits disponibles est
extrêmement large et ceux-ci peuvent servir une variété de fonctions. En
effet, une variété de gabarits professionnels et de haute \emph{qualité
graphique} sont offerts gratuitement en ligne pour des articles, des
livres, des rapports, des \emph{curriculum vitæs} () ou encore des
feuilles de temps pour des contrats rémunérés ().

\emph{Photos de CVs et feuilles de temps professionnels}

Un autre avantage non-négligeable de Markdown--qui le distingue à cet
égard de \LaTeX-\/- est la \emph{flexibilité} qu'il offre à ses
utilisateurs. En effet, en utilisant Pandoc Markdown, qui est une
extension du langage Markdown de base permettant de combiner plusieurs
langages de balisage différents en un seul document, il est possible
d'intégrer dans un seul document plusieurs langages de balisage
différents tels que \LaTeX, HTML, CSS, JavaScript et \texttt{R} en
utilisant l'extension RMarkdown.\footnote{RMarkdown est une extension
  qui permet d'intégrer du code \texttt{R} au sein d'un fichier Markdown
  via RStudio, qui est l'interface la plus communément utilisée pour
  travailler avec \texttt{R}.} Ceci permet donc à l'utilisateur de
bénéficier des fonctionnalités de différents langages dans un seul
document, rendant ainsi possible une variété de \emph{personalisations}
qui ne seraient pas possible autrement. Qui plus est, il est aussi
important de noter que Markdown permet de créer des fichiers Word
réguliers, PDF professionnels et HTML à partir d'un même document.
L'utilisateur peut donc choisir à sa convenance et à tout moment de
quelle manière sera compilée le document rédigé. Cette fonctionnalité
est particulièrement pratique dans le cadre de collaboration avec des
chercheurs n'utilisant pas les langages de balisage ainsi que lors de
l'envoie de manuscripts à des revues scientifiques puisque certaines
d'entre elles exigent de recevoir ceux-ci sous forme de document Word.

La facilité avec laquelle peuvent être intégrés et gérés les figures et
tableaux dans des documents \LaTeX~et Markdown a déjà été abordée, mais
il est important de souligner que l'utilisation de l'extension RMarkdown
permet d'ajouter une couche supplémentaire d'intégration. En effet,
RMarkdown permet de créer une figure grâce à du code \texttt{R}, ainsi
que d'intégrer celle-ci au texte et la formatter en un seul document.
Cela se fait grâce à l'intégration de \emph{R code chunks} dans le
document. Le code est produit dans le \emph{chunk} et la figure ou le
tableau qui en résulte apparait dans le document RMarkdown \emph{et} sur
le document fini. La différence entre RMarkdown et \LaTeX~est que ce
dernier ne peut pas prendre en compte le code \texttt{R} et les figures
et tableaux doivent donc être créées dans un document séparé avant
d'être intégrées dans le document \LaTeX.

Bien que l'apprentissage de \LaTeX~et Markdown puisse être parsemmé de
nombreuses embûches, ces deux langages bénéficient d'une communauté
d'utilisateurs en ligne sur laquelle il est possible de s'appuyer afin
de résoudre tout problème rencontré. Les utilisateurs--particulièrement
les plus expérimentés--sont nombreux à partager leur expérience à leurs
collègues rencontrant des problèmes afin de contribuer à régler ceux-ci.
Cette communauté est présente sur une multitude de sites web, bien que
le point de rencontre principal soit le forum Stack Overflow
(https://stackoverflow.com/), qui est également utilisé pour régler des
problèmes de programmation en R. Une simple recherche sur Google d'un
problème rencontré avec \LaTeX~ou Markdown offrira à l'utilisateurs des
centaines de résultats pertinents afin de l'aider, la plupart de ces
résultats étant probablement des échanges sur Stack Overflow.
L'utilisateur pourra donc filtrer les résultats et observer les
nombreuses solutions envisageables à son problème afin de définir
laquelle est la plus appropriée dans sa situation. Il est important de
noter, toutefois, que cette communauté est nettement plus développée
pour les utilisateurs de \LaTeX~que de Markdown, puisque ce dernier
langage est moins répandu que le premier.

Autre preuve de leur grande \emph{flexibilité} et capacité de
\emph{personnalisation}, certaines manières plutôt spécifiques de
formater le texte sont présentement uniquement disponibles avec
\LaTeX~ou Markdown. C'est le cas de la possibilité de séparer
automatiquement un mot en deux en fin de ligne à l'aide d'un tiret s'il
est suffisamment long, ou encore de la possibilité de permettre que la
dernière ligne d'un paragraphe apparaisse seule en haut d'une page --
ou, à l'inverse, que la première ligne d'un paragraphe apparaisse seule
en bas d'une page. Bien qu'il soit rare que nous ayons absolument besoin
de personnaliser le texte de cette manière, ces possibilités peuvent
s'avérer utile lorsque vous rédiger un texte qui doit se conformer en
tout point à un gabarit spécifique. En effet, certaines revues
scientifiques, maisons d'édition ou universités (dans le cadre de la
rédaction de mémoires et thèses) imposent ce type de gabarit inflexible
et parfois plutôt capricieux. C'est dans ce type de contexte que la
\emph{flexibilité} incomparable de \LaTeX~et Markdown peut s'avérer
utile.

Il est aussi important de noter que plusieurs revues scientifiques
recommandent fortement, voire imposent, de leur soumettre des
manuscripts en format \LaTeX~ou Markdown afin de rendre plus facile pour
leurs éditeurs d'adapter ceux-ci à la mise en page de leur revue.

Finalement, il est important de mentionner en terminant que Markdown et
\LaTeX~sont entièrement gratuits et accessibles aux utilisateurs de tous
les systèmes d'exploitation.

Ceci dit, \LaTeX~et Markdown comportent eux aussi leurs désavantages.

Premièrement, \LaTeX~est difficile à apprendre. Certaines tâches qui
peuvent sembler simples comme l'ajout d'un tableau peuvent nécessiter de
nombreuses lignes de code. De plus, à la moindre erreur de frappe dans
l'utilisation d'une balise, le code risque de planter et de ne pas
produire le document PDF souhaité. C'est ce qu'on appelle une erreur de
compilation. La compilation est le processus par lequel un document
écrit en langage de balisage est transformé en fichier textuel, en
format PDF dans le cas de \LaTeX. Markdown est un langage plus simple à
apprendre, avec des balises plus courtes et intuitives. Il occasionne
donc moins d'erreurs de compilation.

Deuxièmement, \LaTeX~est incompatible avec Word, Pages ou Writer. Pour
transférer un fichier de traitement de texte vers \LaTeX, les balises
doivent être ajoutées manuellement une par une. À l'inverse, pour
transférer un document \LaTeX~vers un fichier de traitement de texte,
les balises doivent être retirées une par une. Il est aussi possible de
copier le texte directement à partir du fichier PDF produit par \LaTeX,
mais les fins de ligne sont interprétées par Word, Pages ou Writer comme
des retours plutôt que des espaces, et les accents sont souvent mal
copiés et doivent être réécrits manuellement. Encore une fois, Markdown
évite ce problème en permettant d'écrire un fichier DOCX à partir du
langage de balisage. Le formatage du fichier DOCX demeure un peu
compliqué cependant et doit être fait à partir du modèle d'un autre
document DOCX formaté tel que souhaité. De plus, les fichiers DOCX ne
peuvent pas être transformés en format Markdown.

Troisièmement, bien que plusieurs fonctions \LaTeX~puissent être
utilisées dans des fichiers Markdown, certaines demeurent incompatibles,
ce qui rend certaines tâches possibles à faire uniquement par \LaTeX.

Quatrièmement, il n'y a aucun suivi des modifications en Markdown ou en
\LaTeX. Pour réviser un travail fait dans l'un de ces deux formats, des
commentaires peuvent être ajoutés sur le PDF avec des logiciels comme
Adobe Reader, mais des commentaires peuvent aussi être faits directement
dans le document \LaTeX~ou Markdown. Ces commentaires sont identifiés à
l'aide de balises et n'apparaissent pas dans le PDF résultant.

Cinquièmement, certaines revues scientifiques refusent les fichiers PDF
et demandent que les soumissions soient faites en format DOCX, justement
pour des raisons de suivi des modifications.

Sixièmement, R Markdown permet de visualiser les résultats d'un code R
directement dans le document, mais dans certains cas cela occasionne des
erreurs.

Septièmement, avec les logiciels de bureau qui permettent d'utiliser
\LaTeX~et Markdown, il est impossible de visualiser le résultat final en
temps réel. La compilation est nécessaire au préalable.

Huitièmement, il n'y a pas de compteur de mots ou de caractères.

Neuvièmement, Markdown et \LaTeX~impliquent la création d'un fichier
.TEX ou .MD en plus d'un .PDF, un .BIB et plusieurs autres. Word ne
nécessite qu'un document .DOCX.

Finalement, les langages de balisage permettent également de créer des
pages Web. Bien que les pages Web puissent être créées à partir de sites
Web comme WordPress, les langages de balisage permettent de produire des
résultats plus personnalisables, plus automatisables et avec une plus
grande qualité graphique également. Ainsi, l'utilisation de fichiers
HTML, JavaScript et CSS, dont le code peut également être intégré dans
un fichier Markdown, tout comme c'est le cas avec les fichiers \LaTeX,
afin de produire des pages Web. HTML peut être appris via des cours sur
Code Academy.

Somme toute, Word n'est pas à antagoniser. Il a ses qualités. Mais

\hypertarget{ruxe9flexion-muxe9thodologique-1}{%
\section{Réflexion
méthodologique}\label{ruxe9flexion-muxe9thodologique-1}}

En pratique, comment utiliser Markdown, \LaTeX~et BibTeX?

\LaTeX~a une syntaxe particulière qui demande un certain temps
d'adaptation. Pour écrire une phrase simple comme celle-ci, la phrase
peut être écrite telle quelle. Par contre, pour mettre un \textbf{mot}
en caractères gras, il faut utiliser la balise suivante:
\texttt{\textbackslash{}textbf\{mot\}}. Pour mettre le
\textcolor{red}{mot} en rouge, la balise est
\texttt{\textbackslash{}textcolor\{red\}\{mot\}}. Pour le mettre en
italique et en note de bas de page\footnote{\emph{mot}}, les balises
\texttt{\textbackslash{}footnote\{\textbackslash{}emph\{mot\}\}} peuvent
être utilisées. Ainsi, des balises peuvent contenir d'autres balises. En
langage \LaTeX, une balise commence toujours par une barre oblique
inversée. Par la suite, le nom de la fonction (\emph{emph},
\emph{textbf}, \emph{textcolor}, etc.) est appelé. Enfin, généralement,
le mot à formater est placé entre accolades (\texttt{\{\}}).

Chaque document \LaTeX~commence par un préambule. Celui-ci présente des
informations telles que la taille des caractères, le type d'article, le
format de mise en page, la police de caractères, l'utilisation
d'en-têtes et de pieds de page, ainsi que l'utilisation de
\emph{packages} \LaTeX~permettant différentes fonctionnalités de
personnalisation du document.

Il n'est pas nécessaire ni souhaitable d'apprendre l'ensemble des
fonctions et des \emph{packages} \LaTeX~qui existent. Au contraire, il
est souvent mieux de commencer par un gabarit de document qui convient
au type de document que vous voulez créer et ensuite de rechercher en
anglais sur Stack Overflow la manière d'ajouter des éléments de
formatage que vous ne connaissez pas (par exemple, \emph{highlight latex
text}). Des gabarits sont disponibles sur Overleaf, au
https://fr.overleaf.com/latex/templates.

Markdown fonctionne de manière similaire à \LaTeX, mais se démarque par
sa plus grande flexibilité et sa syntaxe beaucoup plus légère. Par
contre, Markdown est moins «~user-friendly~», c'est-à-dire qu'il y est
plus difficile de modifier l'aspect visuel d'un document. Tout document
Markdown débute avec un court bloc de syntaxe \texttt{YAML} (acronyme de
\texttt{Yet\ Another\ Markup\ Language}) qui définit les paramètres
généraux du document. Voici un bloc \texttt{YAML} typique:

\begin{verbatim}
---
title: "Les langages de balisage"
subtitle: "Ça change pas le monde, sauf que..."
author:
  - Alexandre Fortier-Chouinard^[University of Toronto]
  - Maxime Blanchard^[McGill University]
output: pdf_document
documentclass: article
bibliography: references.bib
---
\end{verbatim}

Outre le titre, le sous-titre et le nom des auteurs, on y trouve aussi
le gabarit servant à construire l'aspect visuel du chapitre, la manière
dans laquelle il est compilé -- dans ce cas-ci, PDF -- ainsi que le
chemin d'arborescence afin d'accéder au document BibTeX où sont
enregistrées les références utilisées. Il est aussi possible d'y définir
la taille de la police ou encore le gabarit servant à définir le type de
bibliographie qui sera utilisé. De manière particulièrement importante,
c'est l'endroit où sont chargés les \emph{packages} \LaTeX~qui seront
utilisés. En effet, la quasi-totalité des \emph{packages} et fonctions
\LaTeX~sont utilisables dans Markdown, alors que l'inverse n'est pas
vrai. Il est donc possible de personnaliser un document Markdown en
utilisant des \emph{packages} ayant été créés pour \LaTeX.

La syntaxe à utiliser au travers du texte est somme toute plutôt simple.
Pour mettre un ou plusieurs \textbf{mots en gras}, il suffit de les
entourer de deux astériques (\texttt{**mots\ en\ gras**}); pour les
mettre \emph{en italique}, il faut les encadrer d'une seule astérique
(\texttt{*en\ italique*}). Pour définir un titre de section ou de
sous-section, il suffit de mettre des \# devant le titre en question.
Plus vous ajoutez de \#, plus le titre sera petit et plus il sera
considéré à un niveau hiérarchique inférieur dans la structure du texte.
La syntaxe Markdown est donc plus légère que celle de \LaTeX, dans le
but d'en rendre la lecture plus simple pour son utilisateur.

Bien que des gabarits Markdown soient disponibles, ceux-ci sont plus
rares. Ils se trouvent pour la plupart sur GitHub, rendus disponibles
par leur créateur. Cela étant dit, leur personnalisation peut s'avérer
plutôt complexe. En somme, Markdown est particulièrement pratique pour
les documents ne nécessitant pas de respecter un gabarit précis et
réquérant simplement un document d'allure simple et professionnelle.

Pour sa part, BibTeX a une syntaxe relativement simple. D'emblée, les
références BibTeX pour des articles et ouvrages scientifiques sont
disponibles sur Google Scholar. Toutefois, pour citer des sites Web ou
des articles de médias, la référence doit être écrite à la main selon un
format précis. Une bibliographie sur BibTeX peut ressembler à ceci~:

\begin{verbatim}
@book{darwin03,
  address = {London},
  author = {Darwin, Charles},
  publisher = {John Murray},
  title = {{On the Origin of Species by Means of Natural Selection
or the Preservation of Favoured Races in the Struggle for Life}},
  year = {1859}
}

@article{goldfarb96,
  title={The Roots of SGML: A Personal Recollection},
  author={Goldfarb, Charles F},
  journal={Technical communication},
  volume={46},
  number={1},
  pages={75},
  year={1999},
  publisher={Society for Technical Communication}
}
\end{verbatim}

Un fichier BibTeX ne contient rien de plus qu'une série de publications
commençant chacune par la balise \texttt{@} suivie du type d'article --
\emph{article}, \emph{book}, \emph{incollection} pour un chapitre de
livre, \emph{inproceedings} pour une présentation dans une conférence,
\emph{unpublished} pour un article non publié et \emph{online} pour un
site Web sont parmi les plus connus -- et des informations sur la
publication mises entre accolades. La première information entre
accolades est le code de la référence, par exemple \texttt{goldfarb96}.
Dans le fichier \LaTeX, l'auteur doit écrire
\texttt{\textbackslash{}cite\{goldfarb96\}} pour voir dans le document
PDF compilé Goldfarb (1996); le lien est automatiquement cliquable et
renvoie à la notice bibliographique correspondante. L'ordre des
publications dans le document BibTeX a peu d'importance, puisque
\LaTeX~réordonne par défaut la bibliographie en ordre alphabétique.

\hypertarget{trucs-et-astuces-1}{%
\section{Trucs et astuces}\label{trucs-et-astuces-1}}

Où puis-je utiliser ces langages de balisage? Contrairement à Microsoft
Word et Apple Pages, plusieurs options

\hypertarget{logiciels-de-bureau}{%
\subsection{Logiciels de bureau}\label{logiciels-de-bureau}}

\hypertarget{mactex-miktex-et-autres-distributions}{%
\subsubsection{\texorpdfstring{MacTeX, MikTeX et autres distributions
\LaTeX}{MacTeX, MikTeX et autres distributions }}\label{mactex-miktex-et-autres-distributions}}

\hypertarget{rstudio-exemples-pruxe9cuxe9dents-visual-studio-et-autres-logiciels-du-chapitre-5-tous}{%
\subsubsection{RStudio (exemples précédents), Visual Studio et autres
logiciels du chapitre 5
(tous?)}\label{rstudio-exemples-pruxe9cuxe9dents-visual-studio-et-autres-logiciels-du-chapitre-5-tous}}

\hypertarget{logiciels-en-ligne}{%
\subsection{Logiciels en ligne}\label{logiciels-en-ligne}}

\hypertarget{overleaf}{%
\subsubsection{Overleaf}\label{overleaf}}

VS code with LiveShare

Markdown en Overleaf

\hypertarget{ruxe9fuxe9rences}{%
\section{Références}\label{ruxe9fuxe9rences}}

\bookmarksetup{startatroot}

\hypertarget{une-image-vaut-mille-mots}{%
\chapter{Une image vaut mille mots}\label{une-image-vaut-mille-mots}}

Camille Tremblay-Antoine\footnote{Université Laval} Nadjim
Fréchet\footnote{Université de Montréal}

\bookmarksetup{startatroot}

\hypertarget{visualisation-graphique-en-r}{%
\chapter{\texorpdfstring{\textbf{Une image vaut mille mots:}
~Visualisation graphique en
R}{ ~Visualisation graphique en R}}\label{visualisation-graphique-en-r}}

\hypertarget{introduction}{%
\section{Introduction}\label{introduction}}

Une fois les données collectées, nettoyées, traitées et analysées, une
partie centrale du travail d'un scientifique des données est de faire
parler les résultats de ses tests empiriques. Il s'agit alors de trouver
la meilleure manière de rendre l'information digeste pour les experts et
initiés de votre discipline académique ou pour le grand public. La
visualisation graphique des données est donc centrale afin de vulgariser
les résultats d'une recherche empirique.

L'objectif de ce chapitre est d'apprendre aux codeurs débutants les
rudiments de la visualisation graphique en R. Ce chapitre présentera
plus particulièrement les packages R \emph{ggplot2} et \emph{dplyr}
eux-mêmes téléchargeable à partir du package \emph{tidyverse}. Si
\emph{dplyr} permet de préparer les données avant leur visualisation,
\emph{ggplot2} est un package dédié à la production de graphiques. Ce
chapitre présente sa grammaire avec une série d'exemples (Wickham 2009;
Wickham, Çetinkaya-Rundel, and Grolemund 2023).

Ce chapitre est plus technique que théorique et permet aux codeurs
débutants d'en apprendre davantage sur la manière de construire des
graphiques en R avec des données concrètes. Cependant, la question
centrale qui devrait vous guider lorsque vous créez des visualisations
est la suivante: \textbf{Comment opimiser l'intelligibilité des
données?} L'objectif d'un graphique n'est pas seulement d'illustrer les
données. Un bon graphique devrait permettre de vulgariser une
information ou de mettre en saillance un aspect particulier des données.
L'objectif communicationnel devrait toujours être gardé en tête. Les
graphiques en exemple dans ce chapitre sont construits avec les données
de l'Étude Électorale Canadienne de 2019 qui sont facilement
téléchargeables sur leur site\footnote{http://www.ces-eec.ca/}.

La première section de ce chapitre expose les options et packages
également disponibles pour la construction de graphiques en R. La
deuxième section de ce chapitre compare les avantages et inconvénients
de l'utilisation de \emph{ggplot2} par rapport aux autres packages de
visualisation de données qui auront été présentés. La troisième section
de chapitre montre des exemples de graphiques construits avec la
grammaire de \emph{ggplot2} en utilisant les données de l'Étude
électorale canadienne de 2019. Les codes employés pour produire les
graphiques en exemple sont disponibles dans l'annexe de ce livre. Ces
codes reproductibles permettront aux codeurs débutants d'adapter ces
derniers pour leurs propres projets.

\hypertarget{ruxe9flexion-thuxe9orique-2}{%
\section{Réflexion théorique}\label{ruxe9flexion-thuxe9orique-2}}

\hypertarget{les-options-disponibles}{%
\subsection{Les options disponibles}\label{les-options-disponibles}}

De nombreux \emph{packages} ont été développés dans le langage R dans le
but de visualiser des données graphiquement, il devient donc facile de
s'y perdre. Heureusement, les options qui s'offrent à nous se précisent
lorsque l'on s'intéresse à ce qui est le plus utilisé dans la communauté
des codeurs de ce langage de programmation. Les \emph{packages} les plus
utilisés représentent des outils qui ont été substantiellement validés
et améliorés par leurs développeurs, mais aussi par une importante
communauté de codeurs en ligne et de chercheurs universitaires. Trois de
ces options sont présentées dans ce chapitre: les graphiques du
\emph{Base R}, le \emph{package} \emph{Lattice} et le \emph{package}
\emph{ggplot2}. Les avantages et inconvénients respectifs de ces trois
approches pour la création de graphiques sont explicités dans les
sections suivantes.

\hypertarget{avantages-et-inconvuxe9nients-de-base-r}{%
\subsubsection{Avantages et inconvénients de Base
R}\label{avantages-et-inconvuxe9nients-de-base-r}}

Le \emph{Base R} est le langage de base de R et il permet de faire de
nombreuses manipulations statistiques sans avoir à installer de
\emph{packages} au préalable. Le \emph{Base R} permet notamment de
produire des graphiques rapidement. Cela peut être utile pour visualiser
la distribution d'une variable ou pour regarder la relation entre deux
d'entre elles par exemple. Pour produire un graphique avec le langage de
base R, il suffit de faire appel à la fonction \emph{plot()}. Avec la
fonction \emph{plot()}, le codeur peut visualiser la distribution d'une
variable seule en spécifiant l'axe des \emph{x} dans cette dernière. Le
codeur peut également visualiser la relation entre deux variables en
spécifiant à l'intérieur de la fonction celles qui composeront les axes
des \emph{x} et des \emph{y} du graphique. Les fonctions
\emph{barplot(), hist()} ou \emph{boxplot()} disponibles dans le
\emph{Base R} permettent de spécifier le style de graphique souhaité,
qu'on veuille représenter nos données sous forme de diagramme à barre,
d'histogramme ou de diagramme en boîtes (Kabacoff 2022, 119--32).

\begin{Shaded}
\begin{Highlighting}[]
\CommentTok{\# Exemple de graphique avec la fonction barplot() du BaseR}

\FunctionTok{barplot}\NormalTok{(y,}\AttributeTok{names.arg=}\NormalTok{x,}
 \AttributeTok{main=}\StringTok{"Figure 1 {-} Proportion (\%) de répondants par province}\SpecialCharTok{\textbackslash{}n}\StringTok{"}\NormalTok{,}
 \AttributeTok{col =} \StringTok{"blue"}\NormalTok{,}
 \AttributeTok{sub=}\StringTok{"}\SpecialCharTok{\textbackslash{}n}\StringTok{Source: Étude Électorale Canadienne de 2019                                                "}\NormalTok{) }
\end{Highlighting}
\end{Shaded}

Alors qu'un peu tout peut être fait avec le \emph{Base R}, ce langage
demeure élémentaire; il est difficile d'innover dans la visualisation ou
même de produire des graphiques plus sophistiqués. Le \emph{Base R} peut
sembler plus simple pour l'exploration de données ou pour produire des
graphiques de base rapidement, mais ce langage devient rapidement
complexe lorsqu'on cherche à améliorer l'esthétique de son graphique ou
visualiser des relations entre plusieurs variables, ce que
\emph{lattice} et \emph{ggplot2} permettent plus facilement(Wickham
2009, 3--4).

\hypertarget{avantages-et-inconvuxe9nients-de-lattice}{%
\subsubsection{\texorpdfstring{Avantages et inconvénients de
\emph{lattice}}{Avantages et inconvénients de lattice}}\label{avantages-et-inconvuxe9nients-de-lattice}}

Développé par Deepayan Sarkar, \emph{lattice} cherche à faciliter la
visualisation de graphique en facettes. Plus précisément, ce
\emph{package} vise à améliorer les graphiques du \emph{Base R} en
fournissant de meilleures options de graphisme par défaut pour
visualiser des relations multivariées. Ce \emph{package} est donc
intéressant pour les chercheurs et les codeurs voulant présenter
graphiquement la relation entre plus de deux variables (Kabacoff 2022,
373--77; Sarkar 2008, 2023). Pour produire un graphique de base avec
\emph{Lattice}, le \emph{package lattice} doit préalablement être
installé dans la bibliothèque de \emph{packages} du codeur et chargé
dans sa session au début de son code (voir annexe). Par la suite, le
codeur doit spécifier le type de graphique souhaité avec la fonction
appropri/e\footnote{Plusieurs options disponibles comme des histogrammes
  avec la fonction \emph{histogram()} ou des graphiques de densité avec
  la fonction \emph{densityplot()}.}. Une fois la fonction choisie, il
doit spécifier par une formule les variables x et y ainsi que la
troisième variable à contrôler et à visualiser en facettes
(\emph{graph\_type(formula \textbar{} variable en facettes, data=)}).

Si la Figure 1 produite à partir du \emph{Base R} nous permet de
visualiser le pourcentage de répondants par province dans l'Étude
Électorale Canadienne de 2019, le \emph{package lattice} nous permet de
visualiser facilement ce même pourcentage de répondants en tenant compte
du positionnement idéologique des Canadiens par province sur l'échelle
gauche-droite, comme l'illustre la Figure 2 (0 étant la gauche et 10 la
droite).

\begin{Shaded}
\begin{Highlighting}[]
\CommentTok{\# Exemple de graphique avec la fonction histogram() du package lattice}

\FunctionTok{histogram}\NormalTok{(}\SpecialCharTok{\textasciitilde{}}\NormalTok{gaucheDroite }\SpecialCharTok{|}\NormalTok{ province, }\AttributeTok{data =}\NormalTok{ GraphiqueLattice, }\AttributeTok{breaks =} \FunctionTok{seq}\NormalTok{(}\DecValTok{0}\NormalTok{, }\DecValTok{10}\NormalTok{, }
  \AttributeTok{by =} \DecValTok{1}\NormalTok{), }
  \AttributeTok{main =} \StringTok{"Figure 2 {-} Distribution des Canadiens}\SpecialCharTok{\textbackslash{}n}\StringTok{ par province sur l\textquotesingle{}échelle gauche{-}droite}\SpecialCharTok{\textbackslash{}n}\StringTok{"}\NormalTok{,}
  \AttributeTok{xlab =} \StringTok{"}\SpecialCharTok{\textbackslash{}n}\StringTok{Idéologie gauche{-}droite"}\NormalTok{,}
  \AttributeTok{ylab =} \StringTok{"Pourcentage (\%)}\SpecialCharTok{\textbackslash{}n}\StringTok{"}\NormalTok{,}
  \AttributeTok{col  =} \StringTok{"blue"}\NormalTok{,}
  \AttributeTok{sub=}\StringTok{"}\SpecialCharTok{\textbackslash{}n}\StringTok{Source: Étude Électorale Canadienne de 2019                                                "}\NormalTok{)}
\end{Highlighting}
\end{Shaded}

Cependant, le \emph{package lattice} a pour désavantage d'avoir un
modèle formel (une grammaire de graphique) moins compréhensible et
intuitif que celui de \emph{ggplot2} lorsque vient le temps d'améliorer
l'esthétisme des graphiques. De plus, sa plus faible popularité cause
que ce \emph{package} reste moins développé par la communauté de codeurs
de R que ne l'est \emph{ggplot2}. Nous examinons plus en détail la
grammaire de graphique de ce dernier \emph{package} ainsi que ses
avantages et inconvénients dans la prochaine section (Kabacoff 2022,
373--77 et 390; Wickham 2009, 6).

\hypertarget{avantages-et-inconvuxe9nients-de-ggplot2}{%
\subsubsection{Avantages et inconvénients de
ggplot2}\label{avantages-et-inconvuxe9nients-de-ggplot2}}

Développé principalement par Hadley Wickham, \emph{ggplot2} est un
\emph{package R} faisant partie de la collection de \emph{packages} de
\emph{tidyverse}. \emph{Ggplot2} peut être donc utilisé avec les autres
\emph{packages} centraux de \emph{tidyverse}, limitant ainsi de
potentiels conflits entre les fonctions de \emph{packages} qui puissent
être incompatibles avec \emph{ggplot2}. Par exemple, le \emph{package
dplyr} de \emph{tidyverse} est très utile pour analyser, organiser et
préparer vos données à visualiser avec \emph{ggplot2} (Wickham et al.
2019; Wickham, Çetinkaya-Rundel, and Grolemund 2023, 30).

Le principal avantage de \emph{ggplot2} reste sa grammaire qui permet à
l'utilisateur de rensre ses graphiques beaucoup plus visuellement
attrayants en facilitant la personalisation esthétique. Ceci permet de
pousser l'esthétisme de vos graphiques à un très haut niveau par rapport
aux autres \emph{packages} de visualisation graphique disponibles en R.
Les graphiques \emph{ggplot2} se construisent couche par couche, soit
par l'ajout des différents éléments du graphique au fur et à mesure dans
le code du graphique à construire.

La première couche des graphiques \emph{ggplot} est généralement celle
des données et des variables à visualiser. Elle contient plusieurs
éléments fondamentaux qui sont essentiels à chaque graphique. Le premier
élément est la spécification de l'utilisation du \emph{package ggplot2}
qui se fait simplement en appelant la fonCtion \emph{ggplot2()}. Dans
cette fonction, il faut ensuite mentionner quelle est la base de données
(data=) ainsi que la fonction qui sera utilisée pour positionner les
données (aes(). Le positionnement le plus courant est de positionner des
données \emph{x} par rapport à des données \emph{y}, ce qui se fait de
la sorte: aes(x=, y=).

La deuxième couche des graphiques \emph{ggplot2} est celle du
\emph{geom}, qui spécifie le type de graphique souhaité. Les types de
graphiques les plus couramment utilisés avec \emph{ggplot2} sont les
nuages de points (\emph{geom\_point()}), les diagrammes de lignes de
tendances ou de séries chronologiques (\emph{geom\_line()}), les courbes
de densité (\emph{geom\_density()}) ainsi que les graphiques à bandes
(\emph{geom\_bar()}). Mais les possibilités sont infinies (ou presque!)
avec ggplot2 et bien plus de types de graphiques existent.

Les autres couches des graphiques \emph{ggplot2} dépendent souvent du
codeur et des étapes de construction de son graphique\footnote{Les
  étapes (couches) d'un graphique \emph{ggplot2} ne sont pas
  nécessairement dans le même ordre d'un graphique à un autre.} (Wickham
2009, 77 et 89-93). Le reste de ce chapitre présente la grammaire de
\emph{ggplot2} avec un exemple de construction de graphique à bande
présenté couche par couche.

\begin{Shaded}
\begin{Highlighting}[]
\CommentTok{\# Première couche de l\textquotesingle{}exemple de graphique}
\CommentTok{\# ggplot2 (base de données, variables et geom)}

\FunctionTok{ggplot}\NormalTok{(}\AttributeTok{data=}\NormalTok{GraphiqueExemple, }\FunctionTok{aes}\NormalTok{(}\AttributeTok{x=}\NormalTok{province, }\AttributeTok{y=}\NormalTok{prop)) }\SpecialCharTok{+}
  \FunctionTok{geom\_bar}\NormalTok{(}\AttributeTok{stat=}\StringTok{"identity"}\NormalTok{) }
\end{Highlighting}
\end{Shaded}

Tel que mentionné dans le dernier paragraphe, la première étape est de
spécifier la base de données et les variables qu'on souhaite visualiser.
Vous vous souviendrez qu'au début de la section nous avons mentionné la
collection \emph{tidyverse}, et plus spécifiquement le \emph{package
dplyr} qui y est compris. Ce dernier a été utilisé pour
nettoyer/calculer la proportion de répondants par province au préalable,
ce qui nous permet de positionner directement la variable \emph{prop}
dans l'axe y.

\clearpage

\hypertarget{ruxe9fuxe9rences-1}{%
\section{Références}\label{ruxe9fuxe9rences-1}}

\bookmarksetup{startatroot}

\hypertarget{outils-de-recherche-la-quuxeate-infinie-de-loptimisation}{%
\chapter{Outils de recherche: la quête infinie de
l'optimisation}\label{outils-de-recherche-la-quuxeate-infinie-de-loptimisation}}

Il est tout simplement impossible de se lancer dans l'accomplissement
efficace des méthodes proposées dans cet ouvrage sans d'abord se définir
une méthode de travail efficace. Peu importe que l'on travaille seul ou
en équipe, des dossiers bien classés, une arborescence claire et un
stockage sécuritaire sont gages de succès. Un environnement ordonné pour
un travail ordonné.

En science sociale,

Question: Où et quand s'arrêter?

\hypertarget{introduction-du-chapitre}{%
\section{Introduction du chapitre}\label{introduction-du-chapitre}}

\begin{itemize}
\item
  Il existe une multitude d'outils pour aider le chercheur à optimiser
  son temps et son efficacité.
\item
  Ce chapitre présentera 2 types d'outils, et proposera une utilisation
  efficace de ceux-ci:
\end{itemize}

\begin{enumerate}
\def\labelenumi{\arabic{enumi}.}
\tightlist
\item
  Outils de stockage de données (Dropbox, Google Drive, etc.)
\item
  Logiciel de gestion de versions décentralisé
\end{enumerate}

\hypertarget{stockage-de-donnuxe9es}{%
\section{Stockage de données}\label{stockage-de-donnuxe9es}}

\begin{itemize}
\tightlist
\item
  Revue non exhaustive des outils disponibles et aperçu historique;
\item
  Pourquoi choisir Dropbox?

  \begin{itemize}
  \tightlist
  \item
    Avantages
  \item
    Inconvénients
  \item
    Comment l'utiliser efficacement (arborescence, \_sharedFolders,
    etc.)
  \item
    Mais si on veut avoir un suivi de modifications?
  \end{itemize}
\end{itemize}

\hypertarget{logiciel-de-gestion-de-versions-duxe9centralisuxe9}{%
\section{Logiciel de gestion de versions
décentralisé}\label{logiciel-de-gestion-de-versions-duxe9centralisuxe9}}

\begin{itemize}
\tightlist
\item
  Revue non exhaustive des outils disponibles et aperçu historique;
\item
  Pourquoi choisir Git et Github?

  \begin{itemize}
  \tightlist
  \item
    Avantages
  \item
    Inconvénients
  \item
    Comment l'utiliser efficacement (en parallèle à Dropbox, etc.)
  \end{itemize}
\end{itemize}

\hypertarget{conclusion-un-exemple-de-lutilisation-des-outils}{%
\section{Conclusion: un exemple de l'utilisation des
outils}\label{conclusion-un-exemple-de-lutilisation-des-outils}}

\begin{itemize}
\tightlist
\item
  Faire des liens avec l'ensemble du livre!
\end{itemize}

\bookmarksetup{startatroot}

\hypertarget{section-2}{%
\chapter{}\label{section-2}}

\bookmarksetup{startatroot}

\hypertarget{references}{%
\chapter*{References}\label{references}}
\addcontentsline{toc}{chapter}{References}

\markboth{References}{References}

\hypertarget{refs}{}
\begin{CSLReferences}{1}{0}
\leavevmode\vadjust pre{\hypertarget{ref-adcock_collier01}{}}%
Adcock, Robert, and David Collier. 2001. {``Measurement {Validity}: {A
Shared Standard} for {Qualitative} and {Quantitative Research}.''}
\emph{American Political Science Review} 95 (3): 529--46.
\url{https://doi.org/10.1017/S0003055401003100}.

\leavevmode\vadjust pre{\hypertarget{ref-employmentandsocialdevelopmentcanada23}{}}%
Employment and Social Development Canada. 2023. {``Data {Scientist} in
{Canada} \textbar{} {Job} Prospects - {Job Bank}.''} {jobbank.gc.ca}.
August 3, 2023.
\url{http://www.jobbank.gc.ca/explore_career/job_market_report/outlook_occupation_report.xhtml}.

\leavevmode\vadjust pre{\hypertarget{ref-goldfarb96}{}}%
Goldfarb, Charles F. 1996. {``The {Roots} of {SGML} -- {A Personal
Recollection}.''} {sgmlsource.com}. 1996.
\url{http://www.sgmlsource.com/history/roots.htm}.

\leavevmode\vadjust pre{\hypertarget{ref-kabacoff22}{}}%
Kabacoff, Robert. 2022. \emph{R in Action: Data Analysis and Graphics
with {R} and {Tidyverse}}. Third edition. {Shelter Island, NY}: {Manning
Publications}.

\leavevmode\vadjust pre{\hypertarget{ref-morandat_etal12}{}}%
Morandat, Floréal, Brandon Hill, Leo Osvald, and Jan Vitek. 2012.
{``Evaluating the {Design} of the {R Language}.''} In \emph{{ECOOP} 2012
-- {Object-Oriented Programming}: 26th {European Conference}, {Beijing},
{China}, {June} 11-16, 2012. {Proceedings}}, edited by James Noble,
7313:104--31. Lecture {Notes} in {Computer Science}. {Berlin,
Heidelberg}: {Springer Berlin Heidelberg}.
\url{https://doi.org/10.1007/978-3-642-31057-7}.

\leavevmode\vadjust pre{\hypertarget{ref-muenchen11}{}}%
Muenchen, Robert A. 2011. \emph{R for {SAS} and {SPSS} Users}. 2nd ed.
Statistics and Computing. {New York}: {Springer}.

\leavevmode\vadjust pre{\hypertarget{ref-sarkar08}{}}%
Sarkar, Deepayan. 2008. \emph{Lattice: {Multivariate Data Visualization}
with {R}}. {New York, NY}: {Springer New York}.
\url{https://doi.org/10.1007/978-0-387-75969-2}.

\leavevmode\vadjust pre{\hypertarget{ref-sarkar23}{}}%
---------. 2023. {``Trellis {Graphics} for {R}.''}
\url{https://cran.r-project.org/web/packages/lattice/lattice.pdf}.

\leavevmode\vadjust pre{\hypertarget{ref-tippmann15}{}}%
Tippmann, Sylvia. 2015. {``Programming Tools: {Adventures} with {R}.''}
\emph{Nature} 517 (7532): 109--10.
\url{https://doi.org/10.1038/517109a}.

\leavevmode\vadjust pre{\hypertarget{ref-wickham09}{}}%
Wickham, Hadley. 2009. \emph{Ggplot2: {Elegant Graphics} for {Data
Analysis}}. {New York, NY}: {Springer New York}.
\url{https://doi.org/10.1007/978-0-387-98141-3}.

\leavevmode\vadjust pre{\hypertarget{ref-wickham_etal19}{}}%
Wickham, Hadley, Mara Averick, Jennifer Bryan, Winston Chang, Lucy
McGowan, Romain François, Garrett Grolemund, et al. 2019. {``Welcome to
the {Tidyverse}.''} \emph{Journal of Open Source Software} 4 (43): 1686.
\url{https://doi.org/10.21105/joss.01686}.

\leavevmode\vadjust pre{\hypertarget{ref-wickham_etal23}{}}%
Wickham, Hadley, Mine Çetinkaya-Rundel, and Garrett Grolemund. 2023.
\emph{R for Data Science: Import, Tidy, Transform, Visualize, and Model
Data}. Second edition. {Beijing}: {O'Reilly}.

\end{CSLReferences}



\end{document}
